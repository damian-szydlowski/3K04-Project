\documentclass{article}
\usepackage{geometry}[margin=0.25in]
\usepackage{amsmath}
\usepackage{amssymb}
\usepackage{graphicx}
\usepackage{hyperref}
\usepackage{array}
\usepackage{tocloft}
\usepackage{float}
\usepackage{empheq}
\usepackage{paracol}
\usepackage{enumitem}
\usepackage{multirow}
\usepackage[most]{tcolorbox}
\usepackage{titlesec}
\usepackage{listings}
\usepackage{xcolor} % For custom colors
\renewcommand{\lstlistingname}{Code}

\lstset{
    language=Python,
    basicstyle=\ttfamily\small,
    numbers=left,
    numberstyle=\tiny\color{gray},
    keywordstyle=\color{blue},
    stringstyle=\color{red},
    commentstyle=\color{green!50!black},
    showstringspaces=false,
    breaklines=true,
    frame=single,
    frameround=B,
    rulecolor=\color{black},
    backgroundcolor=\color{gray!10}
}

\titleclass{\subsubsubsection}{straight}[\subsection]

\newcounter{subsubsubsection}[subsubsection]
\renewcommand\thesubsubsubsection{\thesubsubsection.\arabic{subsubsubsection}}
\renewcommand\theparagraph{\thesubsubsubsection.\arabic{paragraph}}
\titleformat{\subsubsubsection}
  {\normalfont\normalsize\bfseries}{\thesubsubsubsection}{1em}{}
\titlespacing*{\subsubsubsection}
{0pt}{3.25ex plus 1ex minus .2ex}{1.5ex plus .2ex}

\makeatletter
\renewcommand\paragraph{\@startsection{paragraph}{5}{\z@}%
  {3.25ex \@plus1ex \@minus.2ex}%
  {-1em}%
  {\normalfont\normalsize\bfseries}}
\renewcommand\subparagraph{\@startsection{subparagraph}{6}{\parindent}%
  {3.25ex \@plus1ex \@minus .2ex}%
  {-1em}%
  {\normalfont\normalsize\bfseries}}
\def\toclevel@subsubsubsection{4}
\def\toclevel@paragraph{5}
\def\l@subsubsubsection{\@dottedtocline{4}{7em}{4em}}
\def\l@paragraph{\@dottedtocline{5}{10em}{5em}}

\makeatother

\setcounter{secnumdepth}{4}
\setcounter{tocdepth}{5}

\renewcommand{\arraystretch}{1.5}
\hypersetup{
    colorlinks=true,
    linkcolor=black,
    filecolor=magenta,      
    urlcolor=black,
    pdfpagemode=FullScreen,
    pdfstartview=FitH,
}

\tcbset{colframe=black, colback=white, boxrule=1pt, breakable}

\begin{document}
\pagenumbering{roman}
\urlstyle{same}
%Title page

\begin{titlepage}
    \begin{center}
        \vspace*{1cm}
        \Huge
        3K04 Deliverable 2: Documentation
        \vspace{1cm}\\
        \huge
        Group 33
        \normalsize
        \vfill
        Last Updated: 2025-11-27
    \end{center}
 \end{titlepage}

%------------------------------------------------------------%

\newpage
\tableofcontents

%------------------------------------------------------------%

\newpage
\phantomsection
\listoffigures
\addcontentsline{toc}{section}{List of Figures}

\phantomsection
\listoftables
\addcontentsline{toc}{section}{List of Tables}

%------------------------------------------------------------%

\clearpage
\pagenumbering{arabic}
\newpage
\section{Group Members}
\begin{table}[H]
    \caption{Table of Group Members}
    \centering
    \begin{tabular}{|c|c|c|}
        \hline
        Name        & MacID         & Student Number    \\
        \hline
        Ryan Su     & sur21         & 400507973         \\
        \hline
        Cameron Lin & lin422        & 400535393         \\
        \hline
        Braden McEachern & mceacb1  & 400527617 \\
        \hline
        Damian Szydlowski & szydlowd & 400512629 \\
        \hline
        Menakan Thamilchelvan & thamilcm & 400510755\\ 

        \hline
    \end{tabular}
\end{table}

\newpage
\section{Abbreviations}
\subsection{General Abbreviations}
\begin{enumerate}[label=]
    \item \textbf{BPM} - Beats Per Minute
    \item \textbf{CCS} - Cardiac Conduction System
    \item \textbf{DCM} - Device Controller-Monitor
    \item \textbf{GPIO} - General Purpose Input Output
    \item \textbf{GUI} - Graphical User Interface
    \item \textbf{PWM} - Pulse Width Modulation
\end{enumerate}

\subsection{Bradycardia Operating Abbreviations}
\begin{table}[H]
    \caption{Bradycardia Operating Abbreviations}
    \begin{tabular}{|m{2cm}|m{2cm}|m{2cm}|m{2cm}|m{3cm}|}
        \hline
        \textbf{Category} & \textbf{Chambers Paced} & \textbf{Chambers Sensed} & \textbf{Response to Sensing} & \textbf{Rate Modulation} \\
        \hline 
        Letters     & O-None & O-None & O-None & R-Rate Modulation\\
                    & A-Atrium & A-Atrium & T-Triggered & \\
                    & V-Ventricle & V-Ventricle & I-Inhibited & \\
                    & D-Dual & D-Dual & D-Tracked & \\
        \hline
    \end{tabular}
\end{table}
\newpage

\section{Introduction}

It is hard to understate the importance of the human heart. The heart is the core part of the cardiovascular system; supplying nutrients and oxygen to all the cells
and removing carbon dioxide, especially to vital organs such as the brain, it is imperative for it to be working flawlessly and harmoniously at all times.
Unfortunately, however, cardiovascular diseases are a leading cause of death globally, many of which are caused from complications with 
abnormal heart rhythms. A pacemaker is an implantable device capable of sending timed electrical impulses causing contractions at 
appropriate intervals. Understanding the operation and design of this life saving device will aid in developing 
more efficient and reliable cardiac assistive technology. 

The purpose of this project is to design and implement a system that operates a cardiac pacemaker 
under specified modes. This project will be accomplished through an understanding of embedded systems and through engineering 
principles of software development. 

The scope of this deliverable is to design and implement the embedded pacemaker software, driver software and user interface for 
the DCM while updating and maintaining documentation. 



\section{Requirements}

\subsection{System Requirements}

\begin{table}[H]
\centering
\caption{System Requirements}
\begin{tabular}{|c|p{5cm}|p{8cm}|}
\hline
ID & Requirement & Rationale \\
\hline
SR1 & System Modes &
The system shall implement AOO, VOO, VVI, and AAI pacing modes as the basic bradycardia modes required by the spec. \\
\hline
SR2 & Hardware Hiding &
The system shall use a hardware abstraction layer that maps logical control signals to GPIO pins, which improves maintainability and supports future hardware changes. \\
\hline
SR3 & Simulink and DCM Separation &
The pacing logic shall be implemented in Simulink and the DCM as a separate GUI so that embedded logic and clinical interface can evolve independently. \\
\hline
\end{tabular}
\end{table}


\subsection{Programming Requirements}

\begin{table}[H]
\centering
\caption{Programming Requirements}
\begin{tabular}{|c|p{5cm}|p{8cm}|}
\hline
ID & Requirement & Description / Rationale \\
\hline
PR1 & Programmable Pulse Amplitude &
The pacemaker shall generate atrial and ventricular pacing signals with amplitudes configurable by the user. Adjustable amplitudes allow tuning of pacing strength. \\
\hline
PR2 & Programmable Pulse Width &
The pacemaker shall generate atrial and ventricular signals with pulse widths configurable by the user so timing of stimulation can be adjusted. \\
\hline
PR3 & Programmable Rate Timing &
The pacemaker shall have programmable lower rate limit (LRL) and upper rate limit (URL) that control the minimum and maximum pacing rates, preventing bradycardia and tachycardia. \\
\hline
PR4 & Programmable Refractory Periods &
Atrial and ventricular pulse modes shall implement refractory periods during which sensed events are ignored to avoid double sensing and ringing. \\
\hline
PR5 & Programmable Sensitivity &
A programmable sensitivity threshold for event detection shall be adjustable from the DCM so that sensing can be adapted to patient signals and noise. \\
\hline
PR6 & Pacing Responses &
Each pacing mode shall follow the response implied by its letters: O for asynchronous pacing that ignores sensing, I for inhibited pacing, and T for triggered pacing from sensed pulses. \\
\hline
\end{tabular}
\end{table}


\subsection{Hardware Requirements}

\begin{table}[H]
\centering
\caption{Hardware Requirements}
\begin{tabular}{|c|p{5cm}|p{8cm}|}
\hline
ID & Requirement & Description / Rationale \\
\hline
HR1 & Hardware Hiding Layer &
A hardware abstraction layer shall map digital logic signals to analog front end hardware pins so that the pacing model does not reference physical pins directly, improving readability and portability. \\
\hline
HR2 & Front End Enable Control &
A dedicated signal shall enable or disable the front end sensing circuitry so ADCs and amplifiers can be powered only when needed, reducing noise and power consumption. \\
\hline
\end{tabular}
\end{table}


\subsection{DCM Requirements}

\begin{table}[H]
\centering
    \caption{DCM Requirements}
    \begin{tabular}{|c|p{5cm}|p{8cm}|}
    \hline
    ID & Requirement & Description / Rationale \\
    \hline
    DCM1 & User Authentication &
    The DCM GUI shall provide user registration and login functionality supporting up to ten stored users to control access to the programming interface. \\
    \hline
    DCM2 & Parameter Display and Editing &
    The DCM shall display and allow editing of pacemaker parameters such as LRL, URL, atrial and ventricular amplitude, pulse width, and sensitivity so that clinicians can program the device. \\
    \hline
    DCM3 & Status Indicators &
    The DCM shall provide visible indicators of device connection status and communication loss so the user remains aware of system state and faults. \\
    \hline
    \end{tabular}
\end{table}

\subsection{Serial Requirements}

\begin{table}[H]
    \centering
    \caption{Serial \& Data Requirements}
    \begin{tabular}{|c|p{5cm}|p{8cm}|}
    \hline
    ID & Requirement & Description / Rationale \\
    \hline
    COM1 & The DCM and Pacemaker shall communicate via a defined Serial Protocol (UART). & 
    Establishes communication between the pacemaker and DCM, replacing the mock communication. \\
    \hline
    COM2 & The DCM shall verify the integrity of sent parameters by reading them back (Echo Verification) from the device. & Ensures safety by confirming data was stored correctly. \\
    \hline
    DAT1 & The system shall support programmable Pulse Widths from 1ms to 30ms (1ms increment). & Updated requirements from the previous deliverable. \\
    \hline
    DAT2 & The system shall support programmable Pulse Amplitudes from 0.1V to 5.0V (0.1V increment). & Updated requirements from the previous deliverable. \\
    \hline
    \end{tabular}
\end{table}

\subsection{Egram Requirements}

\begin{table}[H]
    \centering
    \caption{Serial \& Data Requirements}
    \begin{tabular}{|c|p{5cm}|p{8cm}|}
    \hline
    ID & Requirement & Description / Rationale \\
    \hline
    EKG1 & The DCM shall receive real-time electrogram (egram) data points from the pacemaker via Serial. & 
    Allows the doctor to visualize heart activity. \\
    \hline
    EKG2 & The DCM shall plot two simultaneous graphs (Atrial and Ventricular) in real-time. & Required for diagnostic visibility. \\
    \hline
    \end{tabular}
\end{table}

\subsection{Rate Adaptivity Requirements}

\begin{table}[H]
    \centering
    \caption{Rate Adaptivity Requirements}
    \begin{tabular}{|c|p{5cm}|p{8cm}|}
    \hline
    ID & Requirement & Description / Rationale \\
    \hline
    EKG1 & The DCM shall receive real-time electrogram (egram) data points from the pacemaker via Serial. & 
    Allows the doctor to visualize heart activity. \\
    \hline
    EKG2 & The DCM shall plot two simultaneous graphs (Atrial and Ventricular) in real-time. & Required for diagnostic visibility. \\
    \hline
    \end{tabular}
\end{table}

\subsection{Bonus Requirements}

\begin{table}[H]
    \centering
    \caption{Bonus Requirements}
    \begin{tabular}{|c|p{5cm}|p{8cm}|}
    \hline
    ID & Requirement & Description / Rationale \\
    \hline
    BON1 & The system shall implement DDDR mode with a programmable AV Delay. & 
    Mimics natural dual-chamber heart rhythm. \\
    \hline
    BON2 & The system shall inhibit Ventricular pacing (but not Atrial) when the board pushbutton is held. & Simulates a specific failure or diagnostic state. \\
    \hline
    \end{tabular}
\end{table}



\newpage
\section{Design}
\label{dessec}

\subsection{Simulink Design - Deliverable 1}

\subsubsection{Overall Design}

\begin{tcolorbox}
    \begin{figure}[H]
        \includegraphics[width=\textwidth]{SimWholeView.png}
        \caption{Overall Simulink Mapping}
        \label{SimWholeView}
    \end{figure}
\end{tcolorbox}
The Pacemaker architecure can be split up into 4 main modules, input constant variables, 
input monitor variables, stateflow logic for modes, and hardware hiding. \hyperref[SimWholeView]{Figure 1} below shows the overarcing 
workflow of the system:

\newpage
\subsubsection{Input Constant Variables}

\begin{tcolorbox}
    \begin{figure}[H]
        \includegraphics[width=\textwidth]{ConstIn.png}
        \caption{Constant Input Variables}
        \label{ConstIn}
    \end{figure}
\end{tcolorbox}
In the above image, \hyperref[ConstIn]{Figure 2}, we find changeable variables 
relating to pacemaker operation. For general inputs, the changeable variables are:

\begin{itemize}
    \item \textbf{Mode} - Refers to bradycardia operating modes, e.g AOO, VOO, AAI and VVI.
    \item \textbf{Low Rate Interval} - The number of generated pace pulses per minute, converted from a millisecond time period. 
    \item \textbf{Hysteresis Pace} - When enabled, 1, a longer period is waited before pacing after sensing an event to prevent unwanted pacing pulses from ringing from an event. 
    \item \textbf{Hysteresis Interval} - Specifies the time interval waited in the hysteresis mode in milliseconds.
\end{itemize}
The modfiable atrial variables are:

\begin{itemize}
    \item \textbf{Pace Pulse Width} - Changes the width, length of time, of the pace pulse.
    \item \textbf{Pace Pulse Amplitude} - Changes the amplitude, voltage, of the pace pulse.
    \item \textbf{ARP (Atrial Refactory Period)} - The programmed time interval following an atrial event during which time atrial
            events shall not inhibit nor trigger pacing
    \item \textbf{Sense (Sensitivity)} - Determines the minimum value an atrial signal must be to be considered by the pacemaker. 
\end{itemize}
The modifiable ventricular variables are:

\begin{itemize}
    \item \textbf{Pace Pulse Width} - Changes the width, length of time, of the pace pulse.
    \item \textbf{Pace Pulse Amplitude} - Changes the amplitude, voltage, of the pace pulse.
    \item \textbf{VRP (Ventricle Refactory Period)} - The programmed time interval following an ventricle event during which time atrial
            events shall not inhibit nor trigger pacing.
    \item \textbf{Sense (Sensitivity)} - Determines the minimum value a ventricle signal must be to be considered by the pacemaker.
\end{itemize}

\subsubsection{Monitored Input Variables}

\begin{tcolorbox}
    \begin{figure}[H]\label{InMon}
        \includegraphics[width=\textwidth]{InMon.png}
        \caption{Monitored Input Variables}     
    \end{figure}
\end{tcolorbox}
The monitored input variables can be seen in the above \hyperref[InMon]{Figure 3}. These are the atrial and ventricle detection 
variables. The pulses are sensed with through GPIO pins connecting to a board simulating heart conditions. 

\newpage
\subsubsection{Stateflow Modules}
\begin{tcolorbox}
    \begin{figure}[H]\label{StateMod}
        \includegraphics[width=\textwidth]{StateMod.png}
        \caption{Stateflow Modules}   
    \end{figure}
\end{tcolorbox}
The stateflow diagram in Simulink is shown above. It shows the transition between each mode given the mode input shown in 
\hyperref[ConstIn]{Figure 2}. When switching between modes, a core state is returned to that resets variables to their nominal values. 

\newpage
\subsubsection{AOO Stateflow Model}
\begin{tcolorbox}
    \begin{figure}[H]\label{AOOSF}
        \includegraphics[width=\textwidth]{AOO.png}
        \caption{AOO Stateflow Model}     
    \end{figure}
\end{tcolorbox}
The above stateflow, \hyperref[AOOSF]{Figure 5}, shows the stateflow for the AOO mode. It sets values 
controlling discharge and charging of the capacitor to specified nominal values.

\newpage
\subsubsection{VOO Stateflow Model}
\begin{tcolorbox}
    \begin{figure}[H]\label{VOOSF}
        \includegraphics[width=\textwidth]{VOO.png}
        \caption{VOO Stateflow Model}        
    \end{figure}
\end{tcolorbox}
Similar to the AOO stateflow, the above figure, \hyperref[VOOSF]{Figure 6}, shows the states of 
charging and discharging of the capacitor using specified nominal values. 

\newpage
\subsubsection{VII Stateflow Model}
\begin{tcolorbox}
    \begin{figure}[H]\label{VVISF}
        \includegraphics[width=\textwidth]{VVI.png}
        \caption{VVI Stateflow Model}       
    \end{figure}
\end{tcolorbox}
The above stateflow, \hyperref[VVISF]{Figure 7} shows the FSM for the VVI model. The initial state, pVRP, is the state 
that occurs right after the pacemaker delivers a ventricle pacing pulse. A refactory period occurs during this time, where the 
pacemaker ignores sensor inputs to prevent sensing of its own produced pulse or electrical ringing. After a certain amount of time,
the pReady state is transitioned to where sensor inputs are allowed. This allows for the pacemaker to detect natural heart rhythms
and correct heart rhythms accordingly, or solely deliver pacing pulses. The last state before going to the initial state is the 
discharging state where the pacing pulse is produced. 

\newpage
\subsubsection{AAI Stateflow Model}
\begin{tcolorbox}
    \begin{figure}[H]\label{AAISF}
        \includegraphics[width=\textwidth]{AAI.png}
        \caption{AAI Stateflow Model}        
    \end{figure}
\end{tcolorbox}
The above stateflow, \hyperref[AAISF]{Figure 8} shows the process for the AAI mode. The initial state, pARP, is 
the state occurs right after the pacemaker delivers an atrial pacing pulse. During this time, the pacemaker ignores 
sensing events to prevent it from sensing its own delivered pulse. The next state is transitioned to after 
a set time period, the refactory period, if no natural heartbeat is detected after a certain time inverval, the 
discharging state is then transitioned to. However, if a natural heartbeat is detected, another refactory period occurs 
to prevent the pacemaker from sensing ringing. This state then transitions to the discharging state once an appropriate time 
has passed. 

\newpage
\subsubsection{Hardware Hiding}

\begin{tcolorbox}
    \begin{figure}[H]\label{HardHide}
        \includegraphics[width=\textwidth]{HardwareHiding.png}
        \caption{Hardware Hiding of Model}        
    \end{figure}
\end{tcolorbox}

The above image, \hyperref[HardHide]{Figure 9} shows abstraction of the GPIO pin functions. 
It connects logical output signals to the pins on the FRDM-K64F. This allows for better readability,
thus making the code easier to maintain, debug and safer. 

\newpage
\subsection{Simulink Design - Deliverable 2}

\subsubsection{General Overview}

The below image shows a high-level overview of the Simulink model with all subsystems. The newly added modules include 
the serial communication and rate adaptive logic modules for the newly added serial communication with DCM and pacing modes. 
The input monitored variables now also includes accelerometer values for adaptive rates based on accelerometer values. 

\begin{tcolorbox}
    \begin{figure}[H]
        \centering 
        \includegraphics[width=\textwidth]{simwholeview2.png}
        \caption{General Overview of Simulink Model - Deliverable 2}
    \end{figure}
\end{tcolorbox}

\subsubsection{Serial Communication Module}

The serial communication module can be seen in the figure below. The communication input block is stateflow 
logic that handles how the rxdata being transmitted from the serial recieve block is being processed and unpacked and 
the echo function. The communication output is how data is outputted from the pacemaker back to the DCM via the send\_params function packing all data 
back and sending via the serial transmit block. For simplicity, we chose to use a stateflow logic simuilar to that of tutorial 
3. 

\begin{tcolorbox}
    \begin{figure}[H]
        \centering 
        \includegraphics[width=0.9\textwidth]{sercomsim.png}
        \caption{Serial Communication Module Simulink - Deliverable 2}
    \end{figure}
\end{tcolorbox}

\begin{tcolorbox}
    \begin{figure}[H]
        \centering 
        \includegraphics[width=0.9\textwidth]{comminput.png}
        \caption{Communication Input Block - Deliverable 2}
    \end{figure}
\end{tcolorbox}

\begin{tcolorbox}
    \begin{figure}[H]
        \centering 
        \includegraphics[width=0.9\textwidth]{sendparams.png}
        \caption{Send Parameters Block - Deliverable 2}
    \end{figure}
\end{tcolorbox}

\newpage
\subsubsection{Rate Response}

\subsubsubsection{All Modes}

The requirements for the newly added modes were fulfilled and can be seen in the below images. 
Building off the stateflow logic in deliverable 1, the addition of modes was easily added to the default reset state 
block. 

\begin{tcolorbox}
    \begin{figure}[H]
        \centering 
        \includegraphics[width=0.85\textwidth]{allmodes.png}
        \caption{All Modes and Default State Stateflow Diagram - Deliverable 2}
    \end{figure}
\end{tcolorbox}

\subsubsubsection{Mode Stateflow}
Below are the stateflows for each rate adaptive modes:

\begin{tcolorbox}
    \begin{figure}[H]
        \centering 
        \includegraphics[width=0.6\textwidth]{AOOR.png}
        \caption{AOOR Stateflow - Deliverable 2}
    \end{figure}
\end{tcolorbox}

\begin{tcolorbox}
    \begin{figure}[H]
        \centering 
        \includegraphics[width=0.6\textwidth]{VOOR.png}
        \caption{VOOR Stateflow - Deliverable 2}
    \end{figure}
\end{tcolorbox}

\begin{tcolorbox}
    \begin{figure}[H]
        \centering 
        \includegraphics[width=0.9\textwidth]{AAIR.png}
        \caption{AAIR Stateflow - Deliverable 2}
    \end{figure}
\end{tcolorbox}

\begin{tcolorbox}
    \begin{figure}[H]
        \centering 
        \includegraphics[width=0.9\textwidth]{VVIR.png}
        \caption{VVIR Stateflow - Deliverable 2}
    \end{figure}
\end{tcolorbox}

The logic of these states are similar to that of the regular modes e.g AOO, VVI except are governed by the 
rate\_adapt\_interval. This changes the effective BPM of pacing. The Simulink calculations for these modes can be seen below:

\begin{tcolorbox}
    \begin{figure}[H]
        \centering 
        \includegraphics[width=\textwidth]{rateadaptlog.png}
        \caption{Rate Adaptive Logic - Deliverable 2}
    \end{figure}
\end{tcolorbox}

It takes in newly added parameters maximum sensor rate, response factor, activity threshold and activity level and 
outputs a rate adaptive interval value with constraints and parameters. The logic for desired pace is shown below:

\begin{tcolorbox}
    \begin{figure}[H]
        \centering 
        \includegraphics[width=\textwidth]{despace.png}
        \caption{Desired Pace Logic - Deliverable 2}
    \end{figure}
\end{tcolorbox}

The saturation dynamic block prevents the desired pace rate from being too low, below the low rate, or being too high, above the max sensor 
rate. This calculates the desired pacing rate which is fed into the rate adaptive interval block which can be seen below:

\begin{tcolorbox}
    \begin{figure}[H]
        \centering 
        \includegraphics[width=\textwidth]{rateadaptblock.png}
        \caption{Rate Adaptive Interval Block - Deliverable 2}
    \end{figure}
\end{tcolorbox}

This block includes more logic within itself that resets the desired pace rates when swapping between 
different rate adaptive modes. The logic of which can be seen in the stateflow logic below:

\begin{tcolorbox}
    \begin{figure}[H]
        \centering 
        \includegraphics[width=0.9\textwidth]{stateflowratedapt.png}
        \caption{Stateflow for Mode Switching of Rate Adaptive Logic - Deliverable 2}
    \end{figure}
\end{tcolorbox}

\begin{tcolorbox}
    \begin{figure}[H]
        \centering 
        \includegraphics[width=0.9\textwidth]{modeswitchrateadap.png}
        \caption{Rate Adaptive Mode Reset When Switching Logic - Deliverable 2}
    \end{figure}
\end{tcolorbox}

The way activity level for the logic above is measured is through the accelerometer. This is 
computed with the following block diagram:

\begin{tcolorbox}
    \begin{figure}[H]
        \centering 
        \includegraphics[width=\textwidth]{accel.png}
        \caption{Accelerometer Reading Block Diagram - Deliverable 2}
    \end{figure}
\end{tcolorbox}

The activity level is calculated by finding the magnitude of x and y values first, then adding the absolute z value, normalizing these vales. 
The reason is because we want to treat moving diagonally on the xy-plane to be the same as moving straight in x or y, since the magnitude is the same.
The z-value is added for the case the patient jumps or walks up stairs. 

The sampling rate of the ventricular and atrial inputs is faster, 0.001s, to produce more accurate values in 
AAI and VVI modes. The accelerometer sampling rate is lower and is unable to sample faster than the value shown,
0.01s. 

\begin{tcolorbox}
    \begin{figure}[H]
        \centering 
        \includegraphics[width=\textwidth]{acceldata.png}
        \caption{Accelerometer and Inputs Sampling Rates - Deliverable 2}
    \end{figure}
\end{tcolorbox}

Furthermore, this calculation uses a moving custom moving average block and the calculation of such can be seen below. This is to 
smooth out the signal from small deviations or noise in the measurements of acceleration. 

\begin{tcolorbox}
    \begin{figure}[H]
        \centering 
        \includegraphics[width=\textwidth]{movavg.png}
        \caption{Custom Moving Average - Deliverable 2}
    \end{figure}
\end{tcolorbox}

\newpage
\subsubsection{Outputs}

The ouputs of the Simulink is handled in the outputs subsystem. The subsystem is shown below:

\begin{tcolorbox}
    \begin{figure}[H]
        \centering 
        \includegraphics[width=\textwidth]{outputs.png}
        \caption{Outputs Block - Deliverable 2}
    \end{figure}
\end{tcolorbox}

\subsubsubsection{Ventricular Inhibiting}

This block includes a push button input that inhibits the ventricular pulsing. If held down, it prevents the 
ventricular part of the pacemaker from pacing. This is a safety feature that allows for quick emergency 
shutoff incase of malfunction. The logic for this block is shown below:

\begin{tcolorbox}
    \begin{figure}[H]
        \centering 
        \includegraphics[width=\textwidth]{ventinhib.png}
        \caption{Ventricular Inhibiting Logic - Deliverable 2}
    \end{figure}
\end{tcolorbox}

\newpage
\subsection{DCM Design}

The design of the DCM has 3 seperate sections; the \textbf{model, view and controller.} This is to allow modularity and seperation of 
different functionalities of the DCM from one another. For example, adding functionality can be as simple as making and includiung another class 
in another file. 

\subsubsection{Model}
The model section handles the applications data and user logic. It does not interact directly with the interface but 
only manages data and enforces user authentication. Below is an explanation of the individual Python files and their 
functionality:

\paragraph{User Model}
\begin{enumerate}[label=]
    \item \textbf{Purpose:} To manage all user account information.
    \item \textbf{Functions:}
    \begin{itemize}
        \item Loads and saves login information to or from the users.json file.
        \item Registers new users, checking for duplicates and the max user limit.
        \item Authenticates logins by verifying user passwords
        \item Provides a method to get the amount of registered users which is displayed by the DCM.
    \end{itemize}
\end{enumerate}

\paragraph{Pacing Model}
\begin{enumerate}[label=]
    \item \textbf{Purpose:} Manages pacing parameters for all users.
    \item \textbf{Functions:}
    \begin{itemize}
        \item Loads and saves settings to or from pacing\_settings.json
        \item The data is structured as a dictionary with settings as the key and value as the value.
        \item Saves a dictionary of parameters for a specific user and pacemaker mode
        \item Retrieves the saved parameters for a user and mode
    \end{itemize}
\end{enumerate}


\paragraph{Egram Model}
\begin{enumerate}[label=]
    \item \textbf{Purpose:} Manages the capture of electrogram data 
    \item \textbf{Functions:}
    \begin{itemize}
        \item Adds a single data point to an internal listed
        \item Returns the list of captured data 
        \item Clears the captured data to start a new session 
    \end{itemize}
\end{enumerate}

\subsubsection{View}
The view section is what the user sees and interacts with. It is built with the customtkinter 
library for a more aesthetically pleasing interface. They do not have any control logic, only 
to display data and send user actions such as button clicks to the controller. Below 
are an explanation of each view:

\paragraph{Login View}
\begin{enumerate}[label=]
    \item \textbf{Purpose:} Provides a screen for users to login 
    \item \textbf{Functions:} 
    \begin{itemize}
        \item Welcome class displays the login form and a register button.
        \item Register class displays the new user registration form and a back button.
        \item When a button is clicked, it collects text from the entry fields and calls 
        a method on the controller that handles the login logic. 
        \item The welcome view calls a function to display the current registered user count.
    \end{itemize}
\end{enumerate}

\paragraph{Main View}
\begin{enumerate}[label=]
    \item \textbf{Purpose:} Provides the main application screen after logging in
    \item \textbf{Functions:} 
    \begin{itemize}
        \item MainFrame is the main menu. It shows buttons for each pacing mode AOO, VOO, AAI and VVI as well 
        as mock connection controls. It also displays the current connection status.
        \item DataEntry is the form for editing parameters. It displays all parameters and handles validation 
        of data such as the input being in the correct range and is a numeric type before sending the data to the controller.
    \end{itemize}
\end{enumerate}

\subsubsection{Controller}

\paragraph{Controller}
\begin{enumerate}[label=]
    \item \textbf{Purpose:} The main application class that uses instances of the models and view frames. 
    \item \textbf{Functions:} 
    \begin{itemize}
        \item It initializes itself and creates all view frames, UserModel and PacingModel. It then stores these in a dictionary. 
        \item The navigation is handled by a show\_frame function which requests a frame from the dictionary 
        and uses tkinters library function, tkraise(), to bring it into view. 
        \item Event handling is seperated into handling registration and saving settings. For 
        For registration it takes the username and password from the view, calls the UserModel to verify them, 
        then shows a message to confirm or deny entry. The saving of settings takes mode and data from the DataEntry view 
        and passes them to the PacingModel to be saved. 
    \end{itemize}
\end{enumerate}

\paragraph{Main}
\begin{enumerate}[label=]
    \item \textbf{Purpose:} Its purpose and sole function
    is to create an instance of the controller and run it in the main loop, starting the customtkinter application. 
\end{enumerate}

\subsubsection{Views of DCM GUI}
\begin{tcolorbox}
    \begin{figure}[H]
        \centering
        \includegraphics[width=0.9\textwidth]{loginscreen.png}
        \caption{Login Screen of DCM}
    \end{figure}
\end{tcolorbox}

\begin{tcolorbox}
    \begin{figure}[H]
        \centering
        \includegraphics[width=0.9\textwidth]{mainmenu.png}
        \caption{Main Menu of DCM}
    \end{figure}
\end{tcolorbox}

\begin{tcolorbox}
    \begin{figure}[H]
        \centering
        \includegraphics[width=0.9\textwidth]{genparams.png}
        \caption{Data Entry View}
    \end{figure}
\end{tcolorbox}

\begin{tcolorbox}
    \begin{figure}[H]
        \centering
        \includegraphics[width=0.9\textwidth]{switchindevice.png}
        \caption{Switch in Device Detection from DCM}
    \end{figure}
\end{tcolorbox}

\newpage
\subsection{DCM \& Serial Design - Deliverable 2}

The DCM was changed up in deliverable 2 to allow for serial communication and more modes to be used. 
This is accompanied by a new model in the structure of the DCM, serial\_comms.py. 

\paragraph{Serial\_Comms}
\begin{enumerate}[label=]
    \item \textbf{Purpose:} Provides serial communication functions to the DCM through serial send and recieve functions.
    \item \textbf{Functions:} 
    \begin{itemize}
        \item Serial port management - allows low-level hardware connection by scanning for COM ports through the get\_ports function, establishes connection for the DCM by setting baud rate and other parameters, and safely disconnecting devices. 
        \item Binary Data Structure - Uses struct library to pack and unpack data for more efficient data transfer. 
        \item Send\_params - The main function that allows for sending of parameter data to the pacemaker. It typecasts and scales data to allow for more efficient data transfer. 
        \item Data Verification - The echo feature allows for data to be verified once sent to the pacemaker. 
        \item Data Streaming - The EKG requires real time data streaming to be effective which is handled through this class. 
    \end{itemize}
\end{enumerate}

\subsubsection{Serial Packet Structure}
With the addition of additional parameters a data conversion table is used  
to organize the multiple data types being sent. This can be seen in the table below. The chosen variables 
are due that are sent are due to the fact the increments within each range fit in a Uint8 format. This allows for the 
parameter data to be scaled in code to fit in a Uint8 format then scaled back and typecasted in the Simulink. This datatype choice reduces the amount of 
data being transfered while maintaining the same amount of fidelity. Some parameters such as activity threshold and mode are 
mapped to integers. 

\begin{table}[H]
    \caption{Parameter Settings}
    \begin{tabular}{|p{3.4cm}|p{2.5cm}|p{2.5cm}|p{2.5cm}|p{1cm}|p{1.2cm}|p{1.2cm}|}
        \hline
        \textbf{Parameter} & \textbf{Min} & \textbf{Max} & \textbf{Inc} & \textbf{Unit} & \textbf{Type} & \textbf{Bytes} \\ 
        \hline
        LRL & 30 & 175 & 5 & BPM & Uint8 & 1 \\ 
        \hline
        URL & 50 & 175 & 5 & BPM & Uint8 & 1 \\ 
        \hline
        Max Sensor Rate & 50 & 175 & 5 & BPM & Uint8 & 1 \\ 
        \hline
        Fixed AV Delay & 70 & 300 & 10 & ms & Uint8 & 1 \\ 
        \hline
        Dynamic AV Delay & OFF & ON & N/A & N/A & Bool & 1 \\ 
        \hline
        Sensed AV Delay Offset & 0 \newline (lowest -10) & -100 & -10 & ms & Uint8 & 1 \\ 
        \hline
        Atrial Amplitude & 0.00 & 5.00 & 1.25 & Volts & Uint8 & 1 \\ 
        \hline
        Ventricular Amplitude & 0.00 & 5.00 & 1.25 & Volts & Uint8 & 1 \\ 
        \hline

        Atrial Pulse Width & 0.05 \newline 0.1 & - \newline 1.9 & - \newline 0.1 & ms & Uint8 & 1 \\ 
        \hline
        Ventricular Pulse Width & 0.05 \newline 0.1 & - \newline 1.9 & - \newline 0.1 & ms & Uint8 & 1 \\ 
        \hline
        Atrial Sensitivity & 0.25 \newline 1 & 0.75 \newline 10 & 0.25 \newline 0.5 & mV & Uint8 & 1 \\ 
        \hline
        Ventricular Sensitivity & 0.25 \newline 1 & 0.75 \newline 10 & 0.25 \newline 0.5 & mV & Uint8 & 1 \\ 
        \hline
        VRP & 150 & 500 & 10 & ms & Uint8 & 1 \\ 
        \hline
        ARP & 150 & 500 & 10 & ms & Uint8 & 1 \\ 
        \hline
        PVARP & 150 & 500 & 10 & ms & Uint8 & 1 \\ 
        \hline
        PVARP Extension & 0 & 400 & 50 & ms & Uint8 & 1 \\ 
        \hline
        Hysteresis & OFF & ON & N/A & N/A & Bool & 1 \\ 
        \hline
        Rate Smoothing & 0 & 21 & 3 & N/A & Uint8 & 1 \\ 
        \hline
        % STACKED VALUES for ATR Duration
        ATR Duration & 10 \newline 20 \newline 100 & - \newline 80 \newline 2000 & - \newline 20 \newline 100 & cc & Uint8 & 1 \\ 
        \hline
        ATR Fallback Mode & OFF & ON & N/A & N/A & Bool & 1 \\ 
        \hline
        ATR Fallback Time & 1 & 5 & 1 & min & Uint8 & 1 \\ 
        \hline
        
        Activity Threshold & 
        \multicolumn{3}{p{7.5cm}|}{V-Low, Low, Med-Low, Med, Med-High, High, V-High} & 
        N/A & Uint8 & 1 \\ 
        \hline
        
        Reaction Time & 10 & 50 & 10 & sec & Uint8 & 1 \\ 
        \hline
        Response Factor & 1 & 16 & 1 & N/A & Uint8 & 1 \\ 
        \hline
        Recovery Time & 2 & 16 & 1 & min & Uint8 & 1 \\ 
        \hline
        Mode & 
        \multicolumn{3}{p{7.5cm}|}{OFF, DDD, VDD, DDI, DOO, AOO, AAI, VOO, VVI, AAT, VVT, DDDR, VDDR, DDIR, DOOR, AOOR, AAIR, VOOR, VVIR} & 
        N/A & Uint8 & 1 \\ 
        \hline
    \end{tabular}
\end{table}

\newpage
\begin{tcolorbox}
    \begin{figure}[H]
        \centering 
        \includegraphics[width=\textwidth]{simconv.png}
        \caption{Conversion of Data In Simulink}
    \end{figure}
\end{tcolorbox}

\begin{lstlisting}[caption={Conversion of Data in DCM}]
    def send_params(self, params: dict):
        if not self.ser or not self.ser.is_open: return False
        try:
            data = struct.pack(self.FMT_18_BYTES, 
                               0x16, 0x55, 
                               int(params.get("mode", 0)),
                               int(params.get("a_pw", 0) * 100),
                               int(params.get("v_pw", 0) * 100),
                               int(params.get("lrl", 60)),
                               int(params.get("a_amp", 0) * 10),
                               int(params.get("v_amp", 0) * 10),
                               int(params.get("a_ref", 0) / 10),
                               int(params.get("v_ref", 0) / 10),
                               int(params.get("a_sens", 0) * 10),
                               int(params.get("v_sens", 0) * 10),
                               int(params.get("recov", 0)),
                               int(params.get("resp_fact", 0)),
                               int(params.get("msr", 0)),
                               int(params.get("act_thresh", 0)),
                               int(params.get("react_time", 0)),
                               int(params.get("hyst", 0)))
            self.ser.write(data)
            return True
        except Exception: return False
\end{lstlisting}

\begin{tcolorbox}
    \begin{figure}[H]
        \centering 
        \includegraphics[width=\textwidth]{modemapping.png}
        \caption{Mapping of Modes to Integers In Simulink}
    \end{figure}
\end{tcolorbox}


\newpage 
\subsubsection{Main View Design \& Decisions}
Building on top of the previous design the main page was given plenty of new features. The main page can be seen 
in the screenshot below:

\begin{tcolorbox}
    \begin{figure}[H]
        \centering 
        \includegraphics[width=\textwidth]{dcmmain2.png}
        \caption{DCM Main Page - Deliverable 2}
    \end{figure}
\end{tcolorbox}

Changes can be seen in the top left and right corner, the bottom left corner, and the middle. The top left corner 
now displays COM connections with the computer rather than a mock device. 

The top left features font size changing buttons, this was added for accessibility features as 
a previous issue was that the font size can be too small or too large when the DCM 
window was shurnk or expanded. 

The bottom left corner now shows all COM connections on the PC, selecting the pacemaking device 
is typically identified as the J-Link Segger and connecting to a device that is not identified 
using these key words, a warning is shown as seen below:

\begin{tcolorbox}
    \begin{figure}[H]
        \centering
        \includegraphics[width=\textwidth]{potentialwrong.png}
        \caption{Warning of Potentially Wrong Device}
    \end{figure}
\end{tcolorbox}

This is done through this code snippet in the controller.py file:

\begin{lstlisting}[caption={Correct Device Check}]
# --- SAFETY CHECK ---
safe_keywords = ["mbed", "OpenSDA", "NXP", "DAPLink", "JLink", "Segger"]

is_likely_safe = any(keyword.lower() in port_name_display.lower() for keyword in safe_keywords)

if not is_likely_safe:
    response = messagebox.askyesno(
        "Potential Wrong Device", 
        f"The device '{port_name_display}' does not look like a pacemaker board.\n\n"
        "Do you want to connect anyway?"
    )
\end{lstlisting}

Connecting anyway will not allow you to send data through as a safety feature for 
unrecognized devices. 

If the pacemaking device is connected however, it will prompt confirming connection before 
establishing serial communication. 

Lastly, the middle section has been expanded to both allow for expansion when font size is increased 
and to support all the new modes alongside the EKG view. 

\subsubsection{Parameter Pages Design \& Decisions}

The parameter page has been changed to account for all new parameters. Previously, we greyed out 
non-used parameters for each mode however, with the increase of parameters the view will get cluttered 
and more complicated to navigate. Thus, only changeable parameters for each mode is shown:

\begin{tcolorbox}
    \begin{figure}[H]
        \centering 
        \includegraphics[width=\textwidth]{AOOex.png}
        \caption{Updated Parameter Page for AOO Deliverable 2}
    \end{figure}
\end{tcolorbox}

This view also contains more buttons for the serial communication side of things. These features are 
shown on the left side of the screen with buttons verify send, send, save and the bar for echo verification. 

The save button saves the parameters locally for convenience so you do not have to remember all parameters and 
set them all again each time you want to modify values. 

The verify send sends a packet of data to the simulink that asks for the parameter values and is sent back. 
This is a safety feature to ensure that the right data and parameters were sent to the simulink and nothing was 
improperly sent or corrupted. 

The send button sends a packet of data containing all the parameters to the simulink. This is in compliance with Table 11. To ensure 
simplicity the non-used values are just replaced with 0.0s. The simulink handling the packet unpacking is shown in below where rxdata and status 
are provided by the serial recieve block:

\begin{tcolorbox}
    \begin{figure}[H]
        \centering
        \includegraphics[width=\textwidth]{serialcomm.png}
        \caption{Serial Communication Simulink - Deliverable 2}
    \end{figure}
\end{tcolorbox}

From this, we see as well how the echo works by sending a second byte of 22 rather than 55. 
The send\_params() function sends all parameters back by MUXing all data and sending it back via a serial transmit block. 

For some pages the format is slightly different such as the rate adaptive modes. As the parameter is a word with an 
approximate number value a drop down menu seen below is used to select. 

\begin{tcolorbox}
    \begin{figure}[H]
        \centering 
        \includegraphics[width=\textwidth]{AAIRview.png}
        \caption{DCM View of AAIR - Deliverable 2}
    \end{figure}
\end{tcolorbox}

This corresponds to numerical values in the code of 1.5 at V-LOW with increments of 0.5 with 
higher levels of activity threshold. These values were done through testing with the use of the 
accelerometer as a nominal value of 1 through stationary action and a max value of 7 was found. 

\subsubsection{EKG View Design \& Decisions}

The EKG is handled through the ekg\_model.py and ekg\_view.py. It takes analog data from the pins relating 
to the atrium and ventricle. It samples the data and plots it on a graph using the matplotlib python library.
This was used as it gave many useful features such as the ability to scroll and zoom on the EKG data. The view for the EKG can be seen 
below:
\begin{tcolorbox}
    \begin{figure}[H]
        \centering
        \includegraphics[width=0.9\textwidth]{ekgview.png}
        \caption{View of EKG Page}
    \end{figure}
\end{tcolorbox}

This page also allows you to isolate whether or not you want to look at just the 
atrial pulses, ventricular pulese or both with functions on the bottom left to zoom, scroll 
save and configure the subplots. The view of the EKG can be seen below with dummy data:

\begin{tcolorbox}
    \begin{figure}[H]
        \centering 
        \includegraphics[width=0.9\textwidth]{ekgdata.png}
        \caption{EKG View with Data}
    \end{figure}
\end{tcolorbox}

\newpage 
\subsection{Assurance Cases}

Some features of safety were covered in the design section of this documentation, however this 
section will be more comprehensive. 

\subsubsection{Clamping of Heart Rate For Rate Adapative Pacing}

\begin{enumerate}[label=]
    \item \textbf{Safety Goal:} For rate adaptive pacing modes, the heart rate must not be lower than the lower rate limit and the max sensor limit. 
    \item \textbf{Rationale:} To prevent the heart from beating too slow, lower than lower rate limit, or beating too fast, higher than max sensor limit, to ensure healthy heart operation. 
    \item \textbf{Evidence:} The clamp is hard locked from via how its data is processed in simulink seen in the images below. However, this can be seen live through testing 
    as shown in the demo. The clamp is used in two spots to ensure redundancy as extra signal processing occurs. 
    
    \begin{tcolorbox}
        \begin{figure}[H]
            \centering 
            \includegraphics[width=\textwidth]{clamphr.png}
            \caption{Clamping of Heart Rate 1}
        \end{figure}
    \end{tcolorbox}

    \begin{tcolorbox}
        \begin{figure}[H]
            \centering 
            \includegraphics[width=\textwidth]{clamphr.png}
            \caption{Clamping of Heart Rate 2}
        \end{figure}
    \end{tcolorbox}
\end{enumerate}

\subsubsection{Echo Parameter Verification}

\begin{enumerate}[label=]
    \item \textbf{Safety Goal:} Ensuring that the data transmitted to the pacemaker by the DCM is correct. 
    \item \textbf{Rationale:} The ability to verify the data sent with an echo feature is imperative to ensure 
    data was not corrupted or mismanaged when being sent to the pacemaker. 
    Without this verification it can potentially cause issues by setting unsafe limits or incorrect parameters causing harm long term. 
    \item \textbf{Evidence:} The below images are of the Simulink sending the parameters back as well as the 
    DCM sending back the sent parameters. 
    \begin{tcolorbox}
        \begin{figure}[H]
            \centering 
            \includegraphics[width=\textwidth]{echosim.png}
            \caption{Echo Stateflow in Simulink}
        \end{figure}
    \end{tcolorbox}

    \begin{tcolorbox}
        \begin{figure}[H]
            \centering 
            \includegraphics[width=0.7\textwidth]{sendparams.png}
            \caption{Send Params Function in Simulink}
        \end{figure}
    \end{tcolorbox}

    \begin{lstlisting}[caption={Echo Function in DCM}]
        def get_echo(self):
        if not self.ser or not self.ser.is_open: return None
        try:
            self.ser.reset_input_buffer()
            self.ser.write(struct.pack(self.FMT_11_BYTES, 0x16, 0x22, 0,0,0,0.0,0))
            resp = self.ser.read(9)
            if len(resp) != 9: return None
            u = struct.unpack('<BBBHf', resp)
            return {"red": u[0], "green": u[1], "blue": u[2], "switch_time": u[3], "off_time": u[4]}
        except Exception: return None
    \end{lstlisting}
\end{enumerate}

\subsubsection{Reset Between Mode Switching}

\begin{enumerate}[label=]
    \item \textbf{Safety Goal:} Prevent unwanted data when switching between pacing modes of the pacemaker.
    \item \textbf{Rationale:} When switching between modes a default state is returned to before switching to a new mode. This 
    mode grounds all values to prevent erratic and unwanted behaviour between pacing such as switching to another mode mid pace 
    which may cause complications
    \item \textbf{Evidence:} This is soft-locked by the stateflow diagram shown below:
    \begin{tcolorbox}
        \begin{figure}[H]
            \centering 
            \includegraphics[width=\textwidth]{defstate.png}
            \caption{Default State in Simulink}
        \end{figure}
    \end{tcolorbox}
\end{enumerate}

\subsubsection{Incorrect Device Handling}
\begin{enumerate}[label=]
    \item \textbf{Safety Goal:} When trying to use an unrecognized device the DCM will not allow serial communication and warn the user. 
    \item \textbf{Rationale:} To prevent the DCM from harming an unrecognized device by sending data and to warn the user that the pacemaker 
    is currently not the device being configured.
    \item \textbf{Evidence:} A warning message pops up when an unrecognized device is connected to with code to check the COM for recognizable identifiers.
    \begin{tcolorbox}
        \begin{figure}[H]
            \centering 
            \includegraphics[width=\textwidth]{potentialwrong.png}
            \caption{Potentially Wrong Device Warning in DCM}
        \end{figure}
    \end{tcolorbox}

    \begin{lstlisting}[caption={Safety Check of Device in Code}]
        # --- SAFETY CHECK ---
        safe_keywords = ["mbed", "OpenSDA", "NXP", "DAPLink", "JLink", "Segger"]
        
        is_likely_safe = any(keyword.lower() in port_name_display.lower() for keyword in safe_keywords)

        if not is_likely_safe:
            response = messagebox.askyesno(
                "Potential Wrong Device", 
                f"The device '{port_name_display}' does not look like a pacemaker board.\n\n"
                "Do you want to connect anyway?"
            )
    \end{lstlisting}
\end{enumerate}


\newpage



\section{Requirements Potential Changes}

\begin{enumerate}[label=]
    \item \textbf{Pacing Modes:} Addition of all pacing modes such as dual chamber modes.

\end{enumerate}


\section{Design Decision Potential Changes}

\begin{enumerate}[label=]
    \item \textbf{Expansion of Libraries:} 
\end{enumerate}

\newpage
\section{Module Description}

Although breifly highlighted in the \hyperref[dessec]{design section} of this documentation. This section will go 
into more detail about the purpose of each module, key functionality, variables, and how each module interacts with 
one another. 

As shown in \hyperref[SimWholeView]{Figure 1}, there are 3 primary modules operating the pacemaker; \textbf{input constant variables, 
stateflow logic for modes, and hardware hiding.}

\subsection{Input Constant Variables Module}
This module is highlighted in the \hyperref[dessec]{design section}. It goes through all the state variables that 
are changeable parameters of the pacemaker. These inputs include general inputs such as the pacing mode, low rate interval, and hysteresis settings, as well 
as parameters for atrial and ventricle pacing. These parameters are then used in the stateflow logic module to 
complete pacing to user specifications.

\subsection{Stateflow Logic}
This module has many submodules which are also covered in the \hyperref[dessec]{design section}. The overall state machine controlling 
mode selection and reseting of variables to prevent unwanted pacing behaviour is shown in \hyperref[StateMod]{Figure 4}. This module 
converts input parameters into raw data that can be further converted to electrical signals. This module recieves data 
from the input constant variables module as well as from the monitored input variables. This module then feeds into 
hardware hiding. 

\subsection{Hardware Hiding}
Hardware hiding is also briefly covered in the \hyperref[dessec]{design section}. The purpose of this module is 
to convert the signals from the stateflow logic module into outputted electrical signals through pins. As shown in 
\hyperref[HardHide]{Figure 9}, pins are mapped to certain electrical signals such as atrial outputs, ventricle outputs, 
front end signals, and general pin configurations such as grounding pins. This is designed to ensure coding and modules 
are easier to debug with higher level identification and function of each GPIO. 

\newpage
\section{Testing}

\subsection{SimuLink Mode Testing}

\subsubsection{Testing of AOO}

\begin{enumerate}[label=]
   \item \textbf{Purpose:} The purpose of this test is to test basic AOO functionality.
   \item \textbf{Input Conditions:} General inputs of mode = 0, AOO, and hysteresis = 0, off, and standard 
   atrial inputs. 
   \item \textbf{Expected Output:} Consistent, evenly spaced pulses in the output with disregard for natural heart beats.
   \item \textbf{Actual Output:} Output of testing is exactly that of expected, as shown below in \hyperref[AOOtest]{Figure 10}.
   \item \textbf{Result:} Pass
\end{enumerate}

\begin{tcolorbox}
    \begin{figure}[H]        
        \label{AOOtest}
        \includegraphics[width=\textwidth]{AOOtest1.png}
        \caption{AOO Test}
    \end{figure}
\end{tcolorbox}

\newpage
\hyperref[AOOpulse]{Figure 11} below shows a zoomed in view of one of the pulses in the AOO test above, \hyperref[AOOtest]{Figure 10}. 

\begin{tcolorbox}
    \begin{figure}[H]
        \label{AOOpulse}
        \includegraphics[width=\textwidth]{AOOPulseClose.png}
        \caption{Close-up of AOO Pulse}
    \end{figure}
\end{tcolorbox}

\newpage
\subsubsection{Testing of VOO}

\begin{enumerate}[label=]
   \item \textbf{Purpose:} The purpose of this test is to test basic VOO functionality.
   \item \textbf{Input Conditions:} General inputs of mode = 1, VOO, and hysteresis = 0, off, and standard 
   ventricle inputs. 
   \item \textbf{Expected Output:} Consistent, evenly spaced pulses in the output with disregard for natural heart beats.
   \item \textbf{Actual Output:} Output of testing is exactly that of expected, as shown below in \hyperref[VOOtest]{Figure 12}.
   \item \textbf{Result:} Pass
\end{enumerate}

\begin{tcolorbox}
    \begin{figure}[H]
        \label{VOOtest}
        \includegraphics[width=\textwidth]{VOOtest.png}
        \caption{VOO Test}
    \end{figure}
\end{tcolorbox}

\newpage
\hyperref[VOOpulseclose]{Figure 13} below shows a zoomed in view of one of the pulses in the VOO test above, \hyperref[VOOtest]{Figure 12}

\begin{tcolorbox}
    \begin{figure}[H]
        \label{VOOpulseclose}
        \includegraphics[width=\textwidth]{VOOpulseclose.png}
        \caption{Close-up of VOO Pulse}
    \end{figure}
\end{tcolorbox}

\newpage
\subsubsection{Testing of AAI}

\paragraph{No Natural Heart Rate}

\begin{enumerate}[label=]
   \item \textbf{Purpose:} The purpose of this test is to test basic AAI functionality with no natural heart rhythms.
   \item \textbf{Input Conditions:} General inputs of mode = 2, AAI, and hysteresis = 0, off, and standard 
   artial inputs. 
   \item \textbf{Expected Output:} Consistent, evenly spaced pulses in the output.
   \item \textbf{Actual Output:} Output of testing is exactly that of expected, as shown below in \hyperref[AAItestnohr]{Figure 13}.
   \item \textbf{Result:} Pass
\end{enumerate}

\begin{tcolorbox}
    \begin{figure}[H]
        \label{AAItestnohr}
        \includegraphics[width=\textwidth]{AAItestnohr.png}
        \caption{AAI Test No Heart Rate}
    \end{figure}
\end{tcolorbox}

\newpage
\paragraph{Natural Heart Rate of 45 BPM}

\begin{enumerate}[label=]
   \item \textbf{Purpose:} The purpose of this test is to test basic AAI functionality at a low heart rate.
   \item \textbf{Input Conditions:} General inputs of mode = 2, AAI, and hysteresis = 0, off, standard 
   artial inputs, and monitored atrial pulses at 45 BPM.
   \item \textbf{Expected Output:} The pacemaker should pulse after the refactory period expires. An output of blue pulses right before the natural 
   red pulses is expected.
   \item \textbf{Actual Output:} Output of testing is exactly that of expected, as shown below in \hyperref[AAItest45]{Figure 15}.
   \item \textbf{Result:} Pass
\end{enumerate}


\begin{tcolorbox}
    \begin{figure}[H]\label{AAItest45}
        \includegraphics[width=\textwidth]{AAItest45.png}
        \caption{AAI Test 45 BPM}      
    \end{figure}
\end{tcolorbox}

\newpage
\paragraph{Natural Heart Rate of 75 BPM}

\begin{enumerate}[label=]
   \item \textbf{Purpose:} The purpose of this test is to test basic AAI functionality at a nominal heart rate.
   \item \textbf{Input Conditions:} General inputs of mode = 2, AAI, and hysteresis = 0, off, standard 
   artial inputs, and monitored atrial pulses at 75 BPM.
   \item \textbf{Expected Output:} As a nominal heart reate is being inputted, the pacemaker should not be delivering pacing pulses 
   as the simulated heart rate is nominal and healthy.
   \item \textbf{Actual Output:} Output of testing is exactly that of expected, as shown below in \hyperref[AAItest75]{Figure 16}.
   \item \textbf{Result:} Pass
\end{enumerate}

\begin{tcolorbox}
    \begin{figure}[H]\label{AAItest75}
        \includegraphics[width=\textwidth]{AAItest75.png}
        \caption{AAI Test 75 BPM}
    \end{figure}
\end{tcolorbox}

\newpage
\subsubsection{Testing of VVI}

\paragraph{No Natural Heart Rate}

\begin{enumerate}[label=]
   \item \textbf{Purpose:} The purpose of this test is to test basic VVI functionality with no inputted heart rate.
   \item \textbf{Input Conditions:} General inputs of mode = 3, VVI, and hysteresis = 0, off, standard ventricle 
   inputs, and monitored ventricle pulses.
   \item \textbf{Expected Output:} Consistent, evenly spaced pulses in the output.
   \item \textbf{Actual Output:} Output of testing is exactly that of expected, as shown below in \hyperref[VVItestnor]{Figure 17}.
   \item \textbf{Result:} Pass
\end{enumerate}

\begin{tcolorbox}
    \begin{figure}[H]\label{VVItestnor}
        \includegraphics[width=\textwidth]{VVItestnohr.png}
        \caption{VVI Test No Heart Rate}
    \end{figure}
\end{tcolorbox}

\newpage
\paragraph{Natural Heart Rate at 45 BPM}

\begin{enumerate}[label=]
   \item \textbf{Purpose:} The purpose of this test is to test basic VVI functionality with no inputted heart rate.
   \item \textbf{Input Conditions:} General inputs of mode = 3, VVI, and hysteresis = 0, off, standard ventricle 
   inputs, and monitored ventricle pulses.
   \item \textbf{Expected Output:} The pacemaker should pulse after the refactory period expires. An output of blue pulses right before the natural 
   red pulses is expected.
   \item \textbf{Actual Output:} Output of testing is exactly that of expected, as shown below in \hyperref[VVItest45]{Figure 18}.
   \item \textbf{Result:} Pass
\end{enumerate}

\begin{tcolorbox}
    \begin{figure}[H]\label{VVItest45}
        \includegraphics[width=\textwidth]{VVItest35.png}
        \caption{VVI Test 45 BPM}
        
    \end{figure}
\end{tcolorbox}

\newpage
\paragraph{Natural Heart Rate at 75 BPM}

\begin{enumerate}[label=]
   \item \textbf{Purpose:} The purpose of this test is to test basic VVI functionality with no inputted heart rate.
   \item \textbf{Input Conditions:} General inputs of mode = 3, VVI, and hysteresis = 0, off, standard ventricle 
   inputs, and monitored ventricle pulses.
   \item \textbf{Expected Output:} As a nominal heart reate is being inputted, the pacemaker should not be delivering pacing pulses 
   as the simulated heart rate is nominal and healthy.
   \item \textbf{Actual Output:} Output of testing is exactly that of expected, as shown below in \hyperref[VVItest45]{Figure 18}.
   \item \textbf{Result:} Pass
\end{enumerate}

\begin{tcolorbox}
    \begin{figure}[H]\label{VVItest75}
        \includegraphics[width=\textwidth]{VVItest75.png}
        \caption{VVI Test 75 BPM}
    \end{figure}
\end{tcolorbox}

\newpage
\subsubsection{Hysteresis Testing}

\paragraph{Hysteresis Test 1 (60 BPM)}

\begin{enumerate}[label=]
   \item \textbf{Purpose:} The purpose of this test is to test basic hysteresis mode functionality.
   \item \textbf{Input Conditions:} General inputs of mode = 2, AAI, and hysteresis = 1, on, standard atrium 
   inputs, and monitored atrial pulses at 50 BPM.
   \item \textbf{Expected Output:} No output of the pacemaker.
   \item \textbf{Actual Output:} Output of testing is exactly that of expected, as shown below in \hyperref[Hystest1]{Figure 20}.
   \item \textbf{Result:} Pass
\end{enumerate}

\begin{tcolorbox}
    \begin{figure}[H]\label{Hystest1}
        \includegraphics[width=\textwidth]{hystherisistest.png}
        \caption{Hysterisis Test 1}
    \end{figure}
\end{tcolorbox}

\newpage
\paragraph{Hysteresis Test 2 (50 BPM)}

\begin{enumerate}[label=]
   \item \textbf{Purpose:} The purpose of this test is to test basic hysteresis mode functionality.
   \item \textbf{Input Conditions:} General inputs of mode = 2, AAI, and hysteresis = 1, on, standard atrium 
   inputs, and monitored atrial pulses at 50 BPM.
   \item \textbf{Expected Output:} No output of the pacemaker.
   \item \textbf{Actual Output:} Output of testing is exactly that of expected, as shown below in \hyperref[Hystest2]{Figure 21}.
   \item \textbf{Result:} Pass
\end{enumerate}

\begin{tcolorbox}
    \begin{figure}[H]\label{Hystest2}
        \includegraphics[width=\textwidth]{hythersistest2.png}
        \caption{Hysterisis Test 2}
    \end{figure}
\end{tcolorbox}

\newpage
\paragraph{Hysteresis Test 3 (40 BPM)}

\begin{enumerate}[label=]
   \item \textbf{Purpose:} The purpose of this test is to test basic hysteresis mode functionality.
   \item \textbf{Input Conditions:} General inputs of mode = 2, AAI, and hysteresis = 1, on, standard atrium 
   inputs, and monitored atrial pulses at 50 BPM.
   \item \textbf{Expected Output:} Delayed output signals from the pacemaker.
   \item \textbf{Actual Output:} Output of testing is exactly that of expected, as shown below in \hyperref[Hystest3]{Figure 22}.
   \item \textbf{Result:} Pass
\end{enumerate}

\begin{tcolorbox}
    \begin{figure}[H]\label{Hystest3}
        \includegraphics[width=\textwidth]{hytheresistest3.png}
        \caption{Hysterisis Test 3}
    \end{figure}
\end{tcolorbox}

\newpage
\subsection{DCM Testing}

\subsubsection{Login and Registration}

\begin{enumerate}[label=]
   \item \textbf{Purpose:} The purpose of this test is to verify correct storage of newly registered user data 
   and allow login. 
   \item \textbf{Input Conditions:} A random username and password. This will be used again in the login screen to access the DCM. 
   \item \textbf{Expected Output:} A window should pop up notifying the user an account has been registered. The 
   DCM controls should be accessible after the user logs in. 
   \item \textbf{Actual Output:} Dialogue is shown and the file is updated to include new user data. The user is then brought to the 
   \item \textbf{Result:} Pass
\end{enumerate}

\begin{tcolorbox}
    \begin{figure}[H]\label{regtest}
        \includegraphics[width=\textwidth]{registertest.png}
        \caption{Registration Test}
    \end{figure}
\end{tcolorbox}

\begin{tcolorbox}
    \begin{figure}[H]\label{regres}
        \centering
        \includegraphics[width=0.9\textwidth]{registerres.png}
        \caption{Registration Result}
    \end{figure}
\end{tcolorbox}

\begin{tcolorbox}
    \begin{figure}[H]\label{logtest}
        \centering
        \includegraphics[width=0.9\textwidth]{logintest.png}
        \caption{Login Test}
    \end{figure}
\end{tcolorbox}

\begin{tcolorbox}
    \begin{figure}[H]\label{logres}
        \includegraphics[width=\textwidth]{loginres.png}
        \caption{Login Result}
    \end{figure}
\end{tcolorbox}

\newpage
\subsubsection{Parameter Input Validation}
\begin{enumerate}[label=]
   \item \textbf{Purpose:} To enforce numeric types within an allowed range and to ensure the upper rate interval is greater than lower rate interval. 
   \item \textbf{Input Conditions:} Entering a non-numeric, an out of range number and an upper rate interval that is greater than lower rate interval in parameter settings. 
   \item \textbf{Expected Output:} Invalid input dialogue is shown and parameter changes are not saved.
   \item \textbf{Actual Output:}  
   \item \textbf{Result:} Pass
\end{enumerate}


The below figures show the general parameter page with some values inputted as well as the results of putting 
invalid values into parameter page. 

\begin{tcolorbox}
    \begin{figure}[H]\label{genparam}
        \centering
        \includegraphics[width=\textwidth]{genparams.png}
        \caption{DCM Parameter Page with Values Filled}
    \end{figure}
\end{tcolorbox}

\begin{tcolorbox}
    \begin{figure}[H]\label{rangeparam}
        \centering
        \includegraphics[width=0.9\textwidth]{rangeparam.png}
        \caption{DCM Parameter Error When Number Inputted is Out of Range}
    \end{figure}
\end{tcolorbox}


\begin{tcolorbox}
    \begin{figure}[H]\label{numparam}
        \centering
        \includegraphics[width=0.9\textwidth]{numparam.png}
        \caption{DCM Parameter Error When Non-Number is Inputted}
    \end{figure}
\end{tcolorbox}

\begin{tcolorbox}
    \begin{figure}[H]\label{urlirl}
        \centering
        \includegraphics[width=0.9\textwidth]{urlirl.png}
        \caption{DCM Parameter Error When URL is Larger Than LRL}
    \end{figure}
\end{tcolorbox}

\newpage
\subsubsection{Mode Selection and Data Retrieval}
\begin{enumerate}[label=]
   \item \textbf{Purpose:} To test data storage, ensuring proper saving of user data
   \item \textbf{Input Conditions:} Registering account, logging in, and saving parameters.
   \item \textbf{Expected Output:} User data is now found in the associated JSON files.
   \item \textbf{Actual Output:}  Registered user data and parameters are found in their respective JSON files.
   \item \textbf{Result:} Pass
\end{enumerate}

This test was done in conjunction to previous tests except with a different user registered. A user "asd" 
with password "asd" was used for faster log ins. The following images are of the JSON files and the saved parameters 
from the previous test. 

\begin{tcolorbox}
    \begin{figure}[H]\label{savedparams}
        \centering
        \includegraphics[width=0.8\textwidth]{savedparams.png}
        \caption{Stored Parameter File}
    \end{figure}
\end{tcolorbox}

\begin{tcolorbox}
    \begin{figure}[H]\label{saveduser}
        \centering
        \includegraphics[width=0.8\textwidth]{saveduser.png}
        \caption{Stored Users File}
    \end{figure}
\end{tcolorbox}

\subsection{Serial Testing}

\subsection{Rate Adaptive Test}

\newpage
\section{GenAI Usage}

We used a Generative AI assistant to support development of the DCM. It provided starter boilerplate for a 
Tkinter app with a welcome screen, registration and login, JSON storage capped at ten users, which we then 
adapted and tested. We also used it to clarify Python functions and libraries such as Tkinter, JSON, etc. 
and to troubleshoot installing tkinter. We had AI to clarify comments within the code as well. All design 
decisions, requirements, and validation were done by our team, and we reviewed and verified all AI outputs 
before inclusion. 

\end{document}
