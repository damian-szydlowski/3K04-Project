\documentclass{article}
\usepackage{geometry}[margin=0.25in]
\usepackage{amsmath}
\usepackage{amssymb}
\usepackage{graphicx}
\usepackage{hyperref}
\usepackage{array}
\usepackage{tocloft}
\usepackage{float}
\usepackage{empheq}
\usepackage{paracol}
\usepackage{enumitem}
\usepackage{multirow}
\usepackage[most]{tcolorbox}
\usepackage{titlesec}

\titleclass{\subsubsubsection}{straight}[\subsection]

\newcounter{subsubsubsection}[subsubsection]
\renewcommand\thesubsubsubsection{\thesubsubsection.\arabic{subsubsubsection}}
\renewcommand\theparagraph{\thesubsubsubsection.\arabic{paragraph}}
\titleformat{\subsubsubsection}
  {\normalfont\normalsize\bfseries}{\thesubsubsubsection}{1em}{}
\titlespacing*{\subsubsubsection}
{0pt}{3.25ex plus 1ex minus .2ex}{1.5ex plus .2ex}

\makeatletter
\renewcommand\paragraph{\@startsection{paragraph}{5}{\z@}%
  {3.25ex \@plus1ex \@minus.2ex}%
  {-1em}%
  {\normalfont\normalsize\bfseries}}
\renewcommand\subparagraph{\@startsection{subparagraph}{6}{\parindent}%
  {3.25ex \@plus1ex \@minus .2ex}%
  {-1em}%
  {\normalfont\normalsize\bfseries}}
\def\toclevel@subsubsubsection{4}
\def\toclevel@paragraph{5}
\def\l@subsubsubsection{\@dottedtocline{4}{7em}{4em}}
\def\l@paragraph{\@dottedtocline{5}{10em}{5em}}

\makeatother

\setcounter{secnumdepth}{4}
\setcounter{tocdepth}{5}

\renewcommand{\arraystretch}{1.5}
\hypersetup{
    colorlinks=true,
    linkcolor=black,
    filecolor=magenta,      
    urlcolor=black,
    pdfpagemode=FullScreen,
    pdfstartview=FitH,
}

\tcbset{colframe=black, colback=white, boxrule=1pt, breakable}

\begin{document}
\pagenumbering{roman}
\urlstyle{same}
%Title page

\begin{titlepage}
    \begin{center}
        \vspace*{1cm}
        \Huge
        3K04 Deliverable 1: Documentation
        \vspace{1cm}\\
        \huge
        Group 33
        \normalsize
        \vfill
        Last Updated: 2025-10-26
    \end{center}
 \end{titlepage}

%------------------------------------------------------------%

\newpage
\tableofcontents

%------------------------------------------------------------%

\newpage
\phantomsection
\listoffigures
\addcontentsline{toc}{section}{List of Figures}

\phantomsection
\listoftables
\addcontentsline{toc}{section}{List of Tables}

%------------------------------------------------------------%

\clearpage
\pagenumbering{arabic}
\newpage
\section{Group Members}
\begin{table}[H]
    \caption{Table of Group Members}
    \centering
    \begin{tabular}{|c|c|c|}
        \hline
        Name        & MacID         & Student Number    \\
        \hline
        Ryan Su     & sur21         & 400507973         \\
        \hline
        Cameron Lin & lin422        & 400535393         \\
        \hline
        Braden McEachern & mceacb1  & 400527617 \\
        \hline
        Damian Szydlowski & szydlowd & 400512629 \\
        \hline
        Menakan Thamilchelvan & thamilcm & 400510755\\ 
        \hline
        Yash Panchal & & \\
        \hline
        Said Dokmak & & \\
        \hline
        Ishpreet Bal & & \\
        \hline
    \end{tabular}
\end{table}

\newpage
\section{Abbreviations}
\subsection{General Abbreviations}
\begin{enumerate}[label=]
    \item \textbf{BPM} - Beats Per Minute
    \item \textbf{CCS} - Cardiac Conduction System
    \item \textbf{DCM} - Device Controller-Monitor
    \item \textbf{GPIO} - General Purpose Input Output
    \item \textbf{GUI} - Graphical User Interface
    \item \textbf{PWM} - Pulse Width Modulation
\end{enumerate}

\subsection{Bradycardia Operating Abbreviations}
\begin{table}[H]
    \caption{Bradycardia Operating Abbreviations}
    \begin{tabular}{|m{2cm}|m{2cm}|m{2cm}|m{2cm}|m{3cm}|}
        \hline
        \textbf{Category} & \textbf{Chambers Paced} & \textbf{Chambers Sensed} & \textbf{Response to Sensing} & \textbf{Rate Modulation} \\
        \hline 
        Letters     & O-None & O-None & O-None & R-Rate Modulation\\
                    & A-Atrium & A-Atrium & T-Triggered & \\
                    & V-Ventricle & V-Ventricle & I-Inhibited & \\
                    & D-Dual & D-Dual & D-Tracked & \\
        \hline
    \end{tabular}
\end{table}
\newpage
\section{Part 1}

\subsection{Introduction}

It is hard to understate the importance of the human heart. The heart is the core part of the cardiovascular system; supplying nutrients and oxygen to all the cells
and removing carbon dioxide, especially to vital organs such as the brain, it is imperative for it to be working flawlessly and harmoniously at all times.
Unfortunately, however, cardiovascular diseases are a leading cause of death globally, many of which are caused from complications with 
abnormal heart rhythms. A pacemaker is an implantable device capable of sending timed electrical impulses causing contractions at 
appropriate intervals. Understanding the operation and design of this life saving device will aid in developing 
more efficient and reliable cardiac assistive technology. 

The purpose of this project is to design and implement a system that operates a cardiac pacemaker 
under specified modes. This project will be accomplished through an understanding of embedded systems and through engineering 
principles of software development. 

The scope of this deliverable is to design and implement the embedded pacemaker software, driver software and user interface for 
the DCM while updating and maintaining documentation. 



\subsection{Requirements}

\subsubsection{System Requirements}

\begin{table}[H]
\centering
\caption{System Requirements}
\begin{tabular}{|c|p{4cm}|p{7cm}|}
\hline
ID & Requirement & Rationale \\
\hline
SR1 & System Modes &
The system shall implement AOO, VOO, VVI, and AAI pacing modes as the basic bradycardia modes required by the spec. \\
\hline
SR2 & Hardware Hiding &
The system shall use a hardware abstraction layer that maps logical control signals to GPIO pins, which improves maintainability and supports future hardware changes. \\
\hline
SR3 & Simulink and DCM Separation &
The pacing logic shall be implemented in Simulink and the DCM as a separate GUI so that embedded logic and clinical interface can evolve independently. \\
\hline
\end{tabular}
\end{table}


\subsubsection{Programming Requirements}

\begin{table}[H]
\centering
\caption{Programming Requirements}
\begin{tabular}{|c|p{5cm}|p{8cm}|}
\hline
ID & Requirement & Description / Rationale \\
\hline
PR1 & Programmable Pulse Amplitude &
The pacemaker shall generate atrial and ventricular pacing signals with amplitudes configurable by the user. Adjustable amplitudes allow tuning of pacing strength. \\
\hline
PR2 & Programmable Pulse Width &
The pacemaker shall generate atrial and ventricular signals with pulse widths configurable by the user so timing of stimulation can be adjusted. \\
\hline
PR3 & Programmable Rate Timing &
The pacemaker shall have programmable lower rate limit (LRL) and upper rate limit (URL) that control the minimum and maximum pacing rates, preventing bradycardia and tachycardia. \\
\hline
PR4 & Programmable Refractory Periods &
Atrial and ventricular pulse modes shall implement refractory periods during which sensed events are ignored to avoid double sensing and ringing. \\
\hline
PR5 & Programmable Sensitivity &
A programmable sensitivity threshold for event detection shall be adjustable from the DCM so that sensing can be adapted to patient signals and noise. \\
\hline
PR6 & Pacing Responses &
Each pacing mode shall follow the response implied by its letters: O for asynchronous pacing that ignores sensing, I for inhibited pacing, and T for triggered pacing from sensed pulses. \\
\hline
\end{tabular}
\end{table}


\subsubsection{Hardware Requirements}

\begin{table}[H]
\centering
\caption{Hardware Requirements}
\begin{tabular}{|c|p{5cm}|p{8cm}|}
\hline
ID & Requirement & Description / Rationale \\
\hline
HR1 & Hardware Hiding Layer &
A hardware abstraction layer shall map digital logic signals to analog front end hardware pins so that the pacing model does not reference physical pins directly, improving readability and portability. \\
\hline
HR2 & Front End Enable Control &
A dedicated signal shall enable or disable the front end sensing circuitry so ADCs and amplifiers can be powered only when needed, reducing noise and power consumption. \\
\hline
\end{tabular}
\end{table}


\subsubsection{DCM Requirements}

\begin{table}[H]
\centering
\caption{DCM Requirements}
\begin{tabular}{|c|p{5cm}|p{8cm}|}
\hline
ID & Requirement & Description / Rationale \\
\hline
DCM1 & User Authentication &
The DCM GUI shall provide user registration and login functionality supporting up to ten stored users to control access to the programming interface. \\
\hline
DCM2 & Parameter Display and Editing &
The DCM shall display and allow editing of pacemaker parameters such as LRL, URL, atrial and ventricular amplitude, pulse width, and sensitivity so that clinicians can program the device. \\
\hline
DCM3 & Status Indicators &
The DCM shall provide visible indicators of device connection status and communication loss so the user remains aware of system state and faults. \\
\hline
\end{tabular}
\end{table}

\newpage
\subsection{Design}
\label{dessec}

\subsubsection{Simulink Design}

\subsubsubsection{Overall Design}

\begin{tcolorbox}
    \begin{figure}[H]
        \includegraphics[width=\textwidth]{SimWholeView.png}
        \caption{Overall Simulink Mapping}
        \label{SimWholeView}
    \end{figure}
\end{tcolorbox}
The Pacemaker architecure can be split up into 4 main modules, input constant variables, 
input monitor variables, stateflow logic for modes, and hardware hiding. \hyperref[SimWholeView]{Figure 1} below shows the overarcing 
workflow of the system:

\newpage
\subsubsubsection{Input Constant Variables}

\begin{tcolorbox}
    \begin{figure}[H]
        \includegraphics[width=\textwidth]{ConstIn.png}
        \caption{Constant Input Variables}
        \label{ConstIn}
    \end{figure}
\end{tcolorbox}
In the above image, \hyperref[ConstIn]{Figure 2}, we find changeable variables 
relating to pacemaker operation. For general inputs, the changeable variables are:

\begin{itemize}
    \item \textbf{Mode} - Refers to bradycardia operating modes, e.g AOO, VOO, AAI and VVI.
    \item \textbf{Low Rate Interval} - The number of generated pace pulses per minute, converted from a millisecond time period. 
    \item \textbf{Hysteresis Pace} - When enabled, 1, a longer period is waited before pacing after sensing an event to prevent unwanted pacing pulses from ringing from an event. 
    \item \textbf{Hysteresis Interval} - Specifies the time interval waited in the hysteresis mode in milliseconds.
\end{itemize}
The modfiable atrial variables are:

\begin{itemize}
    \item \textbf{Pace Pulse Width} - Changes the width, length of time, of the pace pulse.
    \item \textbf{Pace Pulse Amplitude} - Changes the amplitude, voltage, of the pace pulse.
    \item \textbf{ARP (Atrial Refactory Period)} - The programmed time interval following an atrial event during which time atrial
            events shall not inhibit nor trigger pacing
    \item \textbf{Sense (Sensitivity)} - Determines the minimum value an atrial signal must be to be considered by the pacemaker. 
\end{itemize}
The modifiable ventricular variables are:

\begin{itemize}
    \item \textbf{Pace Pulse Width} - Changes the width, length of time, of the pace pulse.
    \item \textbf{Pace Pulse Amplitude} - Changes the amplitude, voltage, of the pace pulse.
    \item \textbf{VRP (Ventricle Refactory Period)} - The programmed time interval following an ventricle event during which time atrial
            events shall not inhibit nor trigger pacing.
    \item \textbf{Sense (Sensitivity)} - Determines the minimum value a ventricle signal must be to be considered by the pacemaker.
\end{itemize}

\subsubsubsection{Monitored Input Variables}

\begin{tcolorbox}
    \begin{figure}[H]\label{InMon}
        \includegraphics[width=\textwidth]{InMon.png}
        \caption{Monitored Input Variables}     
    \end{figure}
\end{tcolorbox}
The monitored input variables can be seen in the above \hyperref[InMon]{Figure 3}. These are the atrial and ventricle detection 
variables. The pulses are sensed with through GPIO pins connecting to a board simulating heart conditions. 

\newpage
\subsubsubsection{Stateflow Modules}
\begin{tcolorbox}
    \begin{figure}[H]\label{StateMod}
        \includegraphics[width=\textwidth]{StateMod.png}
        \caption{Stateflow Modules}   
    \end{figure}
\end{tcolorbox}
The stateflow diagram in Simulink is shown above. It shows the transition between each mode given the mode input shown in 
\hyperref[ConstIn]{Figure 2}. When switching between modes, a core state is returned to that resets variables to their nominal values. 

\newpage
\subsubsubsection{AOO Stateflow Model}
\begin{tcolorbox}
    \begin{figure}[H]\label{AOOSF}
        \includegraphics[width=\textwidth]{AOO.png}
        \caption{AOO Stateflow Model}     
    \end{figure}
\end{tcolorbox}
The above stateflow, \hyperref[AOOSF]{Figure 5}, shows the stateflow for the AOO mode. It sets values 
controlling discharge and charging of the capacitor to specified nominal values.

\newpage
\subsubsubsection{VOO Stateflow Model}
\begin{tcolorbox}
    \begin{figure}[H]\label{VOOSF}
        \includegraphics[width=\textwidth]{VOO.png}
        \caption{VOO Stateflow Model}        
    \end{figure}
\end{tcolorbox}
Similar to the AOO stateflow, the above figure, \hyperref[VOOSF]{Figure 6}, shows the states of 
charging and discharging of the capacitor using specified nominal values. 

\newpage
\subsubsubsection{VII Stateflow Model}
\begin{tcolorbox}
    \begin{figure}[H]\label{VVISF}
        \includegraphics[width=\textwidth]{VVI.png}
        \caption{VVI Stateflow Model}       
    \end{figure}
\end{tcolorbox}
The above stateflow, \hyperref[VVISF]{Figure 7} shows the FSM for the VVI model. The initial state, pVRP, is the state 
that occurs right after the pacemaker delivers a ventricle pacing pulse. A refactory period occurs during this time, where the 
pacemaker ignores sensor inputs to prevent sensing of its own produced pulse or electrical ringing. After a certain amount of time,
the pReady state is transitioned to where sensor inputs are allowed. This allows for the pacemaker to detect natural heart rhythms
and correct heart rhythms accordingly, or solely deliver pacing pulses. The last state before going to the initial state is the 
discharging state where the pacing pulse is produced. 

\newpage
\subsubsubsection{AAI Stateflow Model}
\begin{tcolorbox}
    \begin{figure}[H]\label{AAISF}
        \includegraphics[width=\textwidth]{AAI.png}
        \caption{AAI Stateflow Model}        
    \end{figure}
\end{tcolorbox}
The above stateflow, \hyperref[AAISF]{Figure 8} shows the process for the AAI mode. The initial state, pARP, is 
the state occurs right after the pacemaker delivers an atrial pacing pulse. During this time, the pacemaker ignores 
sensing events to prevent it from sensing its own delivered pulse. The next state is transitioned to after 
a set time period, the refactory period, if no natural heartbeat is detected after a certain time inverval, the 
discharging state is then transitioned to. However, if a natural heartbeat is detected, another refactory period occurs 
to prevent the pacemaker from sensing ringing. This state then transitions to the discharging state once an appropriate time 
has passed. 

\newpage
\subsubsubsection{Hardware Hiding}

\begin{tcolorbox}
    \begin{figure}[H]\label{HardHide}
        \includegraphics[width=\textwidth]{HardwareHiding.png}
        \caption{Hardware Hiding of Model}        
    \end{figure}
\end{tcolorbox}

The above image, \hyperref[HardHide]{Figure 9} shows abstraction of the GPIO pin functions. 
It connects logical output signals to the pins on the FRDM-K64F. This allows for better readability,
thus making the code easier to maintain, debug and safer. 

\newpage
\subsubsection{DCM Design}

The design of the DCM has 3 seperate sections; the \textbf{model, view and controller.} 

\subsubsubsection{Model}
The model section handles the applications data and user logic. It does not interact directly with the interface but 
only manages data and enforces user authentication. Below is an explanation of the individual Python files and their 
functionality:

\paragraph{User Model}
\begin{enumerate}[label=]
    \item \textbf{Purpose:} To manage all user account information.
    \item \textbf{Functions:}
    \begin{itemize}
        \item Loads and saves login information to or from the users.json file.
        \item Registers new users, checking for duplicates and the max user limit.
        \item Authenticates logins by verifying user passwords
        \item Provides a method to get the amount of registered users which is displayed by the DCM.
    \end{itemize}
\end{enumerate}

\paragraph{Pacing Model}
\begin{enumerate}[label=]
    \item \textbf{Purpose:} Manages pacing parameters for all users.
    \item \textbf{Functions:}
    \begin{itemize}
        \item Loads and saves settings to or from pacing\_settings.json
        \item The data is structured as a dictionary with settings as the key and value as the value.
        \item Saves a dictionary of parameters for a specific user and pacemaker mode
        \item Retrieves the saved parameters for a user and mode
    \end{itemize}
\end{enumerate}


\paragraph{Egram Model}
\begin{enumerate}[label=]
    \item \textbf{Purpose:} Manages the capture of electrogram data 
    \item \textbf{Functions:}
    \begin{itemize}
        \item Adds a single data point to an internal listed
        \item Returns the list of captured data 
        \item Clears the captured data to start a new session 
    \end{itemize}
\end{enumerate}

\subsubsubsection{View}
The view section is what the user sees and interacts with. It is built with the customtkinter 
library for a more aesthetically pleasing interface. They do not have any control logic, only 
to display data and send user actions such as button clicks to the controller. Below 
are an explanation of each view:

\paragraph{Login View}
\begin{enumerate}[label=]
    \item \textbf{Purpose:} Provides a screen for users to login 
    \item \textbf{Functions:} 
    \begin{itemize}
        \item Welcome class displays the login form and a register button.
        \item Register class displays the new user registration form and a back button.
        \item When a button is clicked, it collects text from the entry fields and calls 
        a method on the controller that handles the login logic. 
        \item The welcome view calls a function to display the current registered user count.
    \end{itemize}
\end{enumerate}

\paragraph{Main View}
\begin{enumerate}[label=]
    \item \textbf{Purpose:} Provides the main application screen after logging in
    \item \textbf{Functions:} 
    \begin{itemize}
        \item MainFrame is the main menu. It shows buttons for each pacing mode AOO, VOO, AAI and VVI as well 
        as mock connection controls. It also displays the current connection status.
        \item DataEntry is the form for editing parameters. It displays all parameters and handles validation 
        of data such as the input being in the correct range and is a numeric type before sending the data to the controller.
    \end{itemize}
\end{enumerate}

\subsubsubsection{Controller}

\paragraph{Controller}
\begin{enumerate}[label=]
    \item \textbf{Purpose:} The main application class that uses instances of the models and view frames. 
    \item \textbf{Functions:} 
    \begin{itemize}
        \item It initializes itself and creates all view frames, UserModel and PacingModel. It then stores these in a dictionary. 
        \item The navigation is handled by a show\_frame function which requests a frame from the dictionary 
        and uses tkinters library function, tkraise(), to bring it into view. 
        \item Event handling is seperated into handling registration and saving settings. For 
        For registration it takes the username and password from the view, calls the UserModel to verify them, 
        then shows a message to confirm or deny entry. The saving of settings takes mode and data from the DataEntry view 
        and passes them to the PacingModel to be saved. 
    \end{itemize}
\end{enumerate}

\paragraph{Main}
\begin{enumerate}[label=]
    \item \textbf{Purpose:} Its purpose and sole function
    is to create an instance of the controller and run it in the main loop, starting the customtkinter application. 
\end{enumerate}

\subsubsubsection{Views of DCM GUI}
\begin{tcolorbox}
    \begin{figure}[H]
        \centering
        \includegraphics[width=0.9\textwidth]{loginscreen.png}
        \caption{Login Screen of DCM}
    \end{figure}
\end{tcolorbox}

\begin{tcolorbox}
    \begin{figure}[H]
        \centering
        \includegraphics[width=0.9\textwidth]{mainmenu.png}
        \caption{Main Menu of DCM}
    \end{figure}
\end{tcolorbox}

\begin{tcolorbox}
    \begin{figure}[H]
        \centering
        \includegraphics[width=0.9\textwidth]{genparams.png}
        \caption{Data Entry View}
    \end{figure}
\end{tcolorbox}

\begin{tcolorbox}
    \begin{figure}[H]
        \centering
        \includegraphics[width=0.9\textwidth]{switchindevice.png}
        \caption{Switch in Device Detection from DCM}
    \end{figure}
\end{tcolorbox}

\newpage
\section{Part 2}

\subsection{Requirements Potential Changes}

\begin{enumerate}[label=]
    \item \textbf{Pacing Modes:} Additional pacing modes are to be developed such as AOOR, VOOR, AAIR, and VVIR. 
    \item \textbf{Hardware Communication:} The connection between hardware and GUI may be developed in the next deliverable. 
    The functionality implemented in this deliverable will be used to interface with the board.
    \item \textbf{Parameter Changes:} As additional pacing modes are to be developed, this in turn will bring additional parameters that will 
    need to be added and configured. 
    \item \textbf{Electrogram Functionality:} The electrogram functionality will be used to store and display egram data provided by the 
    simulating board. 
\end{enumerate}


\subsection{Design Decision Potential Changes}

\begin{enumerate}[label=]
    \item \textbf{Expansion of Libraries:} As hardware communication will be implemented in the next iteration, 
    hardware communication libraries such as those implementing serial communication will need to be explored.
    \item \textbf{Expansion of GUI:} To accomodate for more pacing modes, the GUI will need to be modified. Furthermore, 
    for improved quality of life features, more graphics and interactive features may be added. 
    \item \textbf{DCM Architecture:} The DCM system may need to be refactored to implement hardware communication. 
    \item \textbf{Simulink Architecture:} As 4 more additional modes are being added for a total of 8 pacing modes, the 
    Simulink structure may need to be modified to integrate all new features more easily. 
    \item \textbf{Data Verification:} As data will be actively transfered from the DCM to the pacemaker hardware, data integrity further becomes 
    more crucial to maintain. A system to verify parameters are correctly sent and saved will need to be implemented. 
\end{enumerate}

\newpage
\subsection{Module Description}

Although breifly highlighted in the \hyperref[dessec]{design section} of this documentation. This section will go 
into more detail about the purpose of each module, key functionality, variables, and how each module interacts with 
one another. 

As shown in \hyperref[SimWholeView]{Figure 1}, there are 3 primary modules operating the pacemaker; \textbf{input constant variables, 
stateflow logic for modes, and hardware hiding.}

\subsubsection{Input Constant Variables Module}
This module is highlighted in the \hyperref[dessec]{design section}. It goes through all the state variables that 
are changeable parameters of the pacemaker. These inputs include general inputs such as the pacing mode, low rate interval, and hysteresis settings, as well 
as parameters for atrial and ventricle pacing. These parameters are then used in the stateflow logic module to 
complete pacing to user specifications.

\subsubsection{Stateflow Logic}
This module has many submodules which are also covered in the \hyperref[dessec]{design section}. The overall state machine controlling 
mode selection and reseting of variables to prevent unwanted pacing behaviour is shown in \hyperref[StateMod]{Figure 4}. This module 
converts input parameters into raw data that can be further converted to electrical signals. This module recieves data 
from the input constant variables module as well as from the monitored input variables. This module then feeds into 
hardware hiding. 

\subsubsection{Hardware Hiding}
Hardware hiding is also briefly covered in the \hyperref[dessec]{design section}. The purpose of this module is 
to convert the signals from the stateflow logic module into outputted electrical signals through pins. As shown in 
\hyperref[HardHide]{Figure 9}, pins are mapped to certain electrical signals such as atrial outputs, ventricle outputs, 
front end signals, and general pin configurations such as grounding pins. This is designed to ensure coding and modules 
are easier to debug with higher level identification and function of each GPIO. 

\newpage
\subsection{Testing}

\subsubsection{SimuLink Mode Testing}

\subsubsubsection{Testing of AOO}

\begin{enumerate}[label=]
   \item \textbf{Purpose:} The purpose of this test is to test basic AOO functionality.
   \item \textbf{Input Conditions:} General inputs of mode = 0, AOO, and hysteresis = 0, off, and standard 
   atrial inputs. 
   \item \textbf{Expected Output:} Consistent, evenly spaced pulses in the output with disregard for natural heart beats.
   \item \textbf{Actual Output:} Output of testing is exactly that of expected, as shown below in \hyperref[AOOtest]{Figure 10}.
   \item \textbf{Result:} Pass
\end{enumerate}

\begin{tcolorbox}
    \begin{figure}[H]        
        \label{AOOtest}
        \includegraphics[width=\textwidth]{AOOtest1.png}
        \caption{AOO Test}
    \end{figure}
\end{tcolorbox}

\newpage
\hyperref[AOOpulse]{Figure 11} below shows a zoomed in view of one of the pulses in the AOO test above, \hyperref[AOOtest]{Figure 10}. 

\begin{tcolorbox}
    \begin{figure}[H]
        \label{AOOpulse}
        \includegraphics[width=\textwidth]{AOOPulseClose.png}
        \caption{Close-up of AOO Pulse}
    \end{figure}
\end{tcolorbox}

\newpage
\subsubsubsection{Testing of VOO}

\begin{enumerate}[label=]
   \item \textbf{Purpose:} The purpose of this test is to test basic VOO functionality.
   \item \textbf{Input Conditions:} General inputs of mode = 1, VOO, and hysteresis = 0, off, and standard 
   ventricle inputs. 
   \item \textbf{Expected Output:} Consistent, evenly spaced pulses in the output with disregard for natural heart beats.
   \item \textbf{Actual Output:} Output of testing is exactly that of expected, as shown below in \hyperref[VOOtest]{Figure 12}.
   \item \textbf{Result:} Pass
\end{enumerate}

\begin{tcolorbox}
    \begin{figure}[H]
        \label{VOOtest}
        \includegraphics[width=\textwidth]{VOOtest.png}
        \caption{VOO Test}
    \end{figure}
\end{tcolorbox}

\newpage
\hyperref[VOOpulseclose]{Figure 13} below shows a zoomed in view of one of the pulses in the VOO test above, \hyperref[VOOtest]{Figure 12}

\begin{tcolorbox}
    \begin{figure}[H]
        \label{VOOpulseclose}
        \includegraphics[width=\textwidth]{VOOpulseclose.png}
        \caption{Close-up of VOO Pulse}
    \end{figure}
\end{tcolorbox}

\newpage
\subsubsubsection{Testing of AAI}

\paragraph{No Natural Heart Rate}

\begin{enumerate}[label=]
   \item \textbf{Purpose:} The purpose of this test is to test basic AAI functionality with no natural heart rhythms.
   \item \textbf{Input Conditions:} General inputs of mode = 2, AAI, and hysteresis = 0, off, and standard 
   artial inputs. 
   \item \textbf{Expected Output:} Consistent, evenly spaced pulses in the output.
   \item \textbf{Actual Output:} Output of testing is exactly that of expected, as shown below in \hyperref[AAItestnohr]{Figure 13}.
   \item \textbf{Result:} Pass
\end{enumerate}

\begin{tcolorbox}
    \begin{figure}[H]
        \label{AAItestnohr}
        \includegraphics[width=\textwidth]{AAItestnohr.png}
        \caption{AAI Test No Heart Rate}
    \end{figure}
\end{tcolorbox}

\newpage
\paragraph{Natural Heart Rate of 45 BPM}

\begin{enumerate}[label=]
   \item \textbf{Purpose:} The purpose of this test is to test basic AAI functionality at a low heart rate.
   \item \textbf{Input Conditions:} General inputs of mode = 2, AAI, and hysteresis = 0, off, standard 
   artial inputs, and monitored atrial pulses at 45 BPM.
   \item \textbf{Expected Output:} The pacemaker should pulse after the refactory period expires. An output of blue pulses right before the natural 
   red pulses is expected.
   \item \textbf{Actual Output:} Output of testing is exactly that of expected, as shown below in \hyperref[AAItest45]{Figure 15}.
   \item \textbf{Result:} Pass
\end{enumerate}


\begin{tcolorbox}
    \begin{figure}[H]\label{AAItest45}
        \includegraphics[width=\textwidth]{AAItest45.png}
        \caption{AAI Test 45 BPM}      
    \end{figure}
\end{tcolorbox}

\newpage
\paragraph{Natural Heart Rate of 75 BPM}

\begin{enumerate}[label=]
   \item \textbf{Purpose:} The purpose of this test is to test basic AAI functionality at a nominal heart rate.
   \item \textbf{Input Conditions:} General inputs of mode = 2, AAI, and hysteresis = 0, off, standard 
   artial inputs, and monitored atrial pulses at 75 BPM.
   \item \textbf{Expected Output:} As a nominal heart reate is being inputted, the pacemaker should not be delivering pacing pulses 
   as the simulated heart rate is nominal and healthy.
   \item \textbf{Actual Output:} Output of testing is exactly that of expected, as shown below in \hyperref[AAItest75]{Figure 16}.
   \item \textbf{Result:} Pass
\end{enumerate}

\begin{tcolorbox}
    \begin{figure}[H]\label{AAItest75}
        \includegraphics[width=\textwidth]{AAItest75.png}
        \caption{AAI Test 75 BPM}
    \end{figure}
\end{tcolorbox}

\newpage
\subsubsubsection{Testing of VVI}

\paragraph{No Natural Heart Rate}

\begin{enumerate}[label=]
   \item \textbf{Purpose:} The purpose of this test is to test basic VVI functionality with no inputted heart rate.
   \item \textbf{Input Conditions:} General inputs of mode = 3, VVI, and hysteresis = 0, off, standard ventricle 
   inputs, and monitored ventricle pulses.
   \item \textbf{Expected Output:} Consistent, evenly spaced pulses in the output.
   \item \textbf{Actual Output:} Output of testing is exactly that of expected, as shown below in \hyperref[VVItestnor]{Figure 17}.
   \item \textbf{Result:} Pass
\end{enumerate}

\begin{tcolorbox}
    \begin{figure}[H]\label{VVItestnor}
        \includegraphics[width=\textwidth]{VVItestnohr.png}
        \caption{VVI Test No Heart Rate}
    \end{figure}
\end{tcolorbox}

\newpage
\paragraph{Natural Heart Rate at 45 BPM}

\begin{enumerate}[label=]
   \item \textbf{Purpose:} The purpose of this test is to test basic VVI functionality with no inputted heart rate.
   \item \textbf{Input Conditions:} General inputs of mode = 3, VVI, and hysteresis = 0, off, standard ventricle 
   inputs, and monitored ventricle pulses.
   \item \textbf{Expected Output:} The pacemaker should pulse after the refactory period expires. An output of blue pulses right before the natural 
   red pulses is expected.
   \item \textbf{Actual Output:} Output of testing is exactly that of expected, as shown below in \hyperref[VVItest45]{Figure 18}.
   \item \textbf{Result:} Pass
\end{enumerate}

\begin{tcolorbox}
    \begin{figure}[H]\label{VVItest45}
        \includegraphics[width=\textwidth]{VVItest35.png}
        \caption{VVI Test 45 BPM}
        
    \end{figure}
\end{tcolorbox}

\newpage
\paragraph{Natural Heart Rate at 75 BPM}

\begin{enumerate}[label=]
   \item \textbf{Purpose:} The purpose of this test is to test basic VVI functionality with no inputted heart rate.
   \item \textbf{Input Conditions:} General inputs of mode = 3, VVI, and hysteresis = 0, off, standard ventricle 
   inputs, and monitored ventricle pulses.
   \item \textbf{Expected Output:} As a nominal heart reate is being inputted, the pacemaker should not be delivering pacing pulses 
   as the simulated heart rate is nominal and healthy.
   \item \textbf{Actual Output:} Output of testing is exactly that of expected, as shown below in \hyperref[VVItest45]{Figure 18}.
   \item \textbf{Result:} Pass
\end{enumerate}

\begin{tcolorbox}
    \begin{figure}[H]\label{VVItest75}
        \includegraphics[width=\textwidth]{VVItest75.png}
        \caption{VVI Test 75 BPM}
    \end{figure}
\end{tcolorbox}

\newpage
\subsubsubsection{Hysteresis Testing}

\paragraph{Hysteresis Test 1 (60 BPM)}

\begin{enumerate}[label=]
   \item \textbf{Purpose:} The purpose of this test is to test basic hysteresis mode functionality.
   \item \textbf{Input Conditions:} General inputs of mode = 2, AAI, and hysteresis = 1, on, standard atrium 
   inputs, and monitored atrial pulses at 50 BPM.
   \item \textbf{Expected Output:} No output of the pacemaker.
   \item \textbf{Actual Output:} Output of testing is exactly that of expected, as shown below in \hyperref[Hystest1]{Figure 20}.
   \item \textbf{Result:} Pass
\end{enumerate}

\begin{tcolorbox}
    \begin{figure}[H]\label{Hystest1}
        \includegraphics[width=\textwidth]{hystherisistest.png}
        \caption{Hysterisis Test 1}
    \end{figure}
\end{tcolorbox}

\newpage
\paragraph{Hysteresis Test 2 (50 BPM)}

\begin{enumerate}[label=]
   \item \textbf{Purpose:} The purpose of this test is to test basic hysteresis mode functionality.
   \item \textbf{Input Conditions:} General inputs of mode = 2, AAI, and hysteresis = 1, on, standard atrium 
   inputs, and monitored atrial pulses at 50 BPM.
   \item \textbf{Expected Output:} No output of the pacemaker.
   \item \textbf{Actual Output:} Output of testing is exactly that of expected, as shown below in \hyperref[Hystest2]{Figure 21}.
   \item \textbf{Result:} Pass
\end{enumerate}

\begin{tcolorbox}
    \begin{figure}[H]\label{Hystest2}
        \includegraphics[width=\textwidth]{hythersistest2.png}
        \caption{Hysterisis Test 2}
    \end{figure}
\end{tcolorbox}

\newpage
\paragraph{Hysteresis Test 3 (40 BPM)}

\begin{enumerate}[label=]
   \item \textbf{Purpose:} The purpose of this test is to test basic hysteresis mode functionality.
   \item \textbf{Input Conditions:} General inputs of mode = 2, AAI, and hysteresis = 1, on, standard atrium 
   inputs, and monitored atrial pulses at 50 BPM.
   \item \textbf{Expected Output:} Delayed output signals from the pacemaker.
   \item \textbf{Actual Output:} Output of testing is exactly that of expected, as shown below in \hyperref[Hystest3]{Figure 22}.
   \item \textbf{Result:} Pass
\end{enumerate}

\begin{tcolorbox}
    \begin{figure}[H]\label{Hystest3}
        \includegraphics[width=\textwidth]{hytheresistest3.png}
        \caption{Hysterisis Test 3}
    \end{figure}
\end{tcolorbox}

\newpage
\subsubsection{DCM Testing}

\subsubsubsection{Login and Registration}

\begin{enumerate}[label=]
   \item \textbf{Purpose:} The purpose of this test is to verify correct storage of newly registered user data 
   and allow login. 
   \item \textbf{Input Conditions:} A random username and password. This will be used again in the login screen to access the DCM. 
   \item \textbf{Expected Output:} A window should pop up notifying the user an account has been registered. The 
   DCM controls should be accessible after the user logs in. 
   \item \textbf{Actual Output:} Dialogue is shown and the file is updated to include new user data. The user is then brought to the 
   \item \textbf{Result:} Pass
\end{enumerate}

\begin{tcolorbox}
    \begin{figure}[H]\label{regtest}
        \includegraphics[width=\textwidth]{registertest.png}
        \caption{Registration Test}
    \end{figure}
\end{tcolorbox}

\begin{tcolorbox}
    \begin{figure}[H]\label{regres}
        \centering
        \includegraphics[width=0.9\textwidth]{registerres.png}
        \caption{Registration Result}
    \end{figure}
\end{tcolorbox}

\begin{tcolorbox}
    \begin{figure}[H]\label{logtest}
        \centering
        \includegraphics[width=0.9\textwidth]{logintest.png}
        \caption{Login Test}
    \end{figure}
\end{tcolorbox}

\begin{tcolorbox}
    \begin{figure}[H]\label{logres}
        \includegraphics[width=\textwidth]{loginres.png}
        \caption{Login Result}
    \end{figure}
\end{tcolorbox}

\newpage
\subsubsubsection{Parameter Input Validation}
\begin{enumerate}[label=]
   \item \textbf{Purpose:} To enforce numeric types within an allowed range and to ensure the upper rate interval is greater than lower rate interval. 
   \item \textbf{Input Conditions:} Entering a non-numeric, an out of range number and an upper rate interval that is greater than lower rate interval in parameter settings. 
   \item \textbf{Expected Output:} Invalid input dialogue is shown and parameter changes are not saved.
   \item \textbf{Actual Output:}  
   \item \textbf{Result:} Pass
\end{enumerate}


The below figures show the general parameter page with some values inputted as well as the results of putting 
invalid values into parameter page. 

\begin{tcolorbox}
    \begin{figure}[H]\label{genparam}
        \centering
        \includegraphics[width=\textwidth]{genparams.png}
        \caption{DCM Parameter Page with Values Filled}
    \end{figure}
\end{tcolorbox}

\begin{tcolorbox}
    \begin{figure}[H]\label{rangeparam}
        \centering
        \includegraphics[width=0.9\textwidth]{rangeparam.png}
        \caption{DCM Parameter Error When Number Inputted is Out of Range}
    \end{figure}
\end{tcolorbox}


\begin{tcolorbox}
    \begin{figure}[H]\label{numparam}
        \centering
        \includegraphics[width=0.9\textwidth]{numparam.png}
        \caption{DCM Parameter Error When Non-Number is Inputted}
    \end{figure}
\end{tcolorbox}

\begin{tcolorbox}
    \begin{figure}[H]\label{urlirl}
        \centering
        \includegraphics[width=0.9\textwidth]{urlirl.png}
        \caption{DCM Parameter Error When URL is Larger Than LRL}
    \end{figure}
\end{tcolorbox}

\newpage
\subsubsubsection{Mode Selection and Data Retrieval}
\begin{enumerate}[label=]
   \item \textbf{Purpose:} To test data storage, ensuring proper saving of user data
   \item \textbf{Input Conditions:} Registering account, logging in, and saving parameters.
   \item \textbf{Expected Output:} User data is now found in the associated JSON files.
   \item \textbf{Actual Output:}  Registered user data and parameters are found in their respective JSON files.
   \item \textbf{Result:} Pass
\end{enumerate}

This test was done in conjunction to previous tests except with a different user registered. A user "asd" 
with password "asd" was used for faster log ins. The following images are of the JSON files and the saved parameters 
from the previous test. 

\begin{tcolorbox}
    \begin{figure}[H]\label{savedparams}
        \centering
        \includegraphics[width=0.8\textwidth]{savedparams.png}
        \caption{Stored Parameter File}
    \end{figure}
\end{tcolorbox}

\begin{tcolorbox}
    \begin{figure}[H]\label{saveduser}
        \centering
        \includegraphics[width=0.8\textwidth]{saveduser.png}
        \caption{Stored Users File}
    \end{figure}
\end{tcolorbox}



\newpage
\subsection{GenAI Usage}

We used a Generative AI assistant to support development of the DCM. It provided starter boilerplate for a 
Tkinter app with a welcome screen, registration and login, JSON storage capped at ten users, which we then 
adapted and tested. We also used it to clarify Python functions and libraries such as Tkinter, JSON, etc. 
and to troubleshoot installing tkinter. We had AI to clarify comments within the code as well. All design 
decisions, requirements, and validation were done by our team, and we reviewed and verified all AI outputs 
before inclusion. 

\end{document}
